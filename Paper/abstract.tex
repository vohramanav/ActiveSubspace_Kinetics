\section*{Abstract}
Time evolution of species' concentration in a chemical system is tightly coupled
with the underlying mechanism and the governing rate laws for individual reactions.
Often, the rate-controlling parameters cannot be specified precisely 
due to \textit{epistemic} uncertainty, i.e., lack of knowledge and 
sparse calibration data. Computer simulations of such systems must
therefore account for the epistemic uncertainty regarding these parameters
when making predictions. However, complex mechanisms typically involve a large number of
reactions and consequently, a large set of uncertain rate parameters. Propagation of
the uncertainty from parameters to predictions can be remarkably challenging for
such high-dimensional problems. In this study, we focus on developing an efficient
approach for uncertainty propagation from inputs to the model output,
through dimension reduction by computing the
so-called
\textit{active subspace}~\cite{Constantine:2015}. The active subspace
 predominantly captures the variability in the quantity of
interest (QoI) in terms of fewer independent variables. 
However, its computation requires gradient estimation of
the model output with respect to its inputs at multiple points in the
input domain, and can therefore be
remarkably challenging. In this study, we compute the active subspace
for the H$_2$/O$_2$ mechanism that involves 19 chemical reactions,
using an iterative strategy that is expected to be relatively more efficient
than the conventional approach. 
The gradient is estimated using finite difference, regarded as a perturbation-based
approach, and using a linear approximation to the available set of model
evaluations, regarded as a regression-based approach. Since finite difference
requires model evaluations at neighboring points, the perturbation-based approach
is expected to be computationally expensive and relatively more accurate.
The active subspace is first
computed for a 19-dimensional problem wherein only the uncertainty in the pre-exponents 
is considered. In this case, the two approaches are observed to yield consistent results.
However, when computing the active subspace for a 33-dimensional problem involving
uncertain activation energies in addition to the pre-exponents, a trade-off between
accuracy and effort is observed. Specifically, when estimating the gradient using
the regression-based approach, the variability of the QoI (in the subspace),
due to uncertainty in the rate parameters, is under-estimated. Moreover, the
sensitivity towards one of the most important parameter is not captured. Hence,
we carefully recommend using the regression-based approach for estimating the
gradient only in situations involving intensive computations of the model 
(e.g. complex mechanisms involving a large number of reactions and species), and where the
goal is to obtain rough estimates of the statistics of the QoI.
