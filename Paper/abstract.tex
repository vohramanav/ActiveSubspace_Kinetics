\section*{Abstract}
Time evolution of species' concentration in a chemical system is tightly coupled
with the underlying mechanism and the governing rate laws for individual reactions.
Often, the rate-controlling parameters cannot be specified precisely 
due to \textit{epistemic} uncertainty, i.e., lack of knowledge and 
sparse calibration data. Computer simulations of such systems must
therefore account for the epistemic uncertainty regarding these parameters
when making predictions. However, complex mechanisms typically involve a large number of
reactions and consequently, a large set of uncertain rate parameters. Propagation of
the uncertainty from parameters to predictions can be remarkably challenging for
such high-dimensional problems. In this study, we focus on developing an efficient
approach for uncertainty propagation from inputs to the model output,
through dimension reduction by computing the
so-called
\textit{active subspace}. The active subspace
 predominantly captures the variability in the quantity of
interest (QoI) in terms of fewer independent variables. 
However, its computation requires gradient estimation of
the model output with respect to its inputs which can be
remarkably challenging. In this study, we compute the active subspace
for the H$_2$/O$_2$ mechanism that involves 19 chemical reactions,
using two iterative strategies. The first strategy uses finite difference 
to estimate the gradient, whereas, the second strategy approximates the
gradient using a regression-based linear approximation of the model output.
The finite difference approach requires additional model evaluations  and 
is therefore computationally expensive. Both strategies yield consistent
results for the case when only the pre-exponents in the Arrhenius rate laws
for the 19 reactions are considered as uncertain. However, implementation 
to a higher-dimensional problem involving 33 uncertain rate parameters:
pre-exponents and activation energies reveals the trade-off between 
accuracy and effort. Specifically, using the second strategy,
the mean, mode, and the uncertainty
in the QoI is reasonably approximated; the sensitivity towards the most
important rate parameter is not captured. Hence, both approaches are
useful depending upon the complexity of the mechanism and the QoI.