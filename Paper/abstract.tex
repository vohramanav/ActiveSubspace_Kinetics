\section*{Abstract}
Time evolution of species' concentration in a chemical system is tightly coupled
with the underlying mechanism and the governing rate laws for individual reactions.
Oftentimes, the rate-controlling parameters cannot be specified precisely 
due to sparse calibration data and lack of knowledge about them, regarded as
the \textit{epistemic} uncertainty. Computer simulations for such systems must
therefore account for the presence of the epistemic uncertainty in these parameters
when making predictions. However, complex mechanisms typically involve a large number of
reactions and consequently, a large set of uncertain rate parameters. Propagation of
the uncertainty from parameters to predictions can be remarkably challenging for
such high-dimensional problems. In this study, we focus on enabling an efficient
approach for uncertainty propagation from inputs to the model output,
through dimension reduction by computing the
so-called
\textit{active subspace}. The active subspace
 predominantly captures the variability in the quantity of
interest (QoI) in terms of fewer independent variables. 
It is computed using an iterative, gradient-based
approach and a computationally less intensive gradient-free approach for the 
H$_2$/O$_2$ mechanism involving 19 reactions. An initial investigation considers
only the 19 pre-exponents of the Arrhenius rate laws as uncertain. In this case, both
approaches yield consistent results wherein the active subspace is found to be 
1-dimensional. The two approaches are further implemented to a high-dimensional
problem involving 33 uncertain rate-parameters: pre-exponent and the activation
energy. Once again, a 1-dimensional active subspace is obtained, indicating
enormous scope for dimension reduction in such applications. However, the
gradient-free approach though reasonably captured the mean, mode, and the uncertainty
in the QoI; failed to capture the sensitivity towards the most
important parameter according to the gradient-based analysis for the high-dimensional
application.   
