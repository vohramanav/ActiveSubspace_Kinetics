\section*{Abstract}
We focus on efficient quantification of uncertainty in complex chemical
reaction networks with a large number of uncertain parameters.  Parameter
dimension reduction is accomplished by computing an~\emph{active subspace} that
predominantly captures the variability in the quantity of interest (QoI).  In
this study, we compute the active subspace for the H$_2$/O$_2$ mechanism that
involves 19 chemical reactions, using an efficient iterative strategy.  The
active subspace is first computed for a 19-parameter problem wherein only the
uncertainty in the pre-exponents of the individual reaction rates is
considered. This is followed by the analysis of a 33-dimensional case wherein
the activation energies are also considered uncertain.  In both cases, a
one-dimensional active subspace is identified, which indicates
enormous potential for efficient statistical analysis of complex 
chemical systems. In addition, we explore links between active subspaces and
global sensitivity analysis measures, and exploit these links for 
indentification of key contributors to the variability in the model response.

%In this case, the two approaches are observed to yield consistent results.
%However, when computing the active subspace for a 33-dimensional problem
%involving uncertain activation energies in addition to the pre-exponents, a
%trade-off between accuracy and effort is observed. 

%Specifically, when estimating the gradient using
%the regression-based approach, the variability of the QoI (in the subspace),
%due to uncertainty in the rate parameters, is under-estimated. Moreover, the
%sensitivity towards one of the most important parameter is not captured. Hence,
%we carefully recommend using the regression-based approach for estimating the
%gradient only in situations involving intensive computations of the model 
%(e.g. complex mechanisms involving a large number of reactions and species), and where the
%goal is to obtain rough estimates of the statistics of the QoI.
