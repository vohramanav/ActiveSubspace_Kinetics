\section{$\text{H}_2$/$\text{O}_2$ reaction kinetics: high-dimensional case}
\label{sec:app}

For the high-dimensional case, we aim to investigate the impact of the uncertain
pre-exponents ($A_i$'s) as well as the activation energies ($E_{a,i}$'s) on ignition 
delay during the H$_2$/O$_2$ reaction. The $A_i$'s were considered to be uniformly
distributed in the interval, $[0.9A_i^\ast, 1.1A_i^\ast]$. The $E_{a,i}$'s for all
reactions except $\mathcal{R}_6$ -- $\mathcal{R}_9$ and $\mathcal{R}_{13}$ (due to 
zero nominal values for $E_a$)
were considered to be uncertain and uniformly distributed in the interval: 
$[0.99E_{a,i}^\ast, 1.01E_{a,i}^\ast]$. The nominal values, $A_i^\ast$ and $E_{a,i}^\ast$
corresponding to the different reaction rates are provided in~\cite{Yetter:1991}. 

The gradient-based approach discussed earlier in~\ref{sub:grad} was used to compute the
active subspace. Using the iterative procedure, the convergence of the eigenvectors
was examined by tracking $\max(\delta \hat{\mat{W}}_{1,j}^{(i)})$, plotted in 
Figure~\ref{fig:conv_app} (right). In Figure~\ref{fig:conv_app} (left), individual
components of the converged eigenvector are illustrated. The convergence was established using
a $\tau$ value of 0.02. Finite difference was used to compute model gradients at 40 samples in
the input domain. Hence, a total of 1360 model evaluations were required for this purpose.  
%
\begin{figure}[htbp]
 \begin{center}
  \includegraphics[width=0.8\textwidth]{./Figures/eigv10}
\caption{Left: An illustrative comparison of individual components of the converged dominant eigenvector obtained
using the two approaches discussed in~\ref{sub:grad} and~\ref{sub:gradfree}. Right: The quantity,  
$\max(\delta \hat{\mat{W}}_{1,j}^{(i)})$
is plotted for successive iterations to illustrate the convergence behavior.}
\label{fig:conv_app}
\end{center}
\end{figure}
%
 
In Figure~\ref{fig:hd} (left), we illustrate the resulting eigenvalue spectrum. The second eigenvalue is
found to be roughly two orders of magnitude smaller than the first, indicating that the active subspace
is 1-dimensional. This is further confirmed by the SSP plots in Figure~\ref{fig:hd} (right). Specifically
in the case of gradient-based approach, the SSP exhibits a linear variation that is reasonably
captured by its linear fit, $\tilde{G}$. Whereas, the SSP based on the gradient-free approach
is relatively more scattered. Consequently, the linear fit exhibits a discrepancy with the corresponding
fit for the gradient-based SSP. 
%
\begin{figure}[htbp]
 \begin{center}
   \includegraphics[width=0.45\textwidth]{./Figures/eig_33D}
   \includegraphics[width=0.45\textwidth]{./Figures/ssp_33D}
\caption{Left: Eigenvalue spectrum of $\hat{\mat{C}}$. Right: SSP of the computed active subspace, and the
corresponding 1-D linear fit.} 
\label{fig:hd}
\end{center}
\end{figure}
%
We plot the normalized activity scores as well as the total Sobol' indices for the 33 uncertain 
rate-controlling parameters in Figure~\ref{fig:as_33D}. Note that the normalized activity scores
as well as the Sobol' indices
were evaluated using respective 1-dimensional surrogates for the two approaches.
%
\begin{figure}[htbp]
 \begin{center}
  \includegraphics[width=0.8\textwidth]{./Figures/as_33D_new}
\caption{A bar-graph illustrating individual activity scores for the uncertain $A_i$' (left) and $E_{a,i}$'s (right).}
\label{fig:as_33D}
\end{center}
\end{figure}
%

Several useful inferences can be drawn from the above plot. Firstly, the total Sobol indices are found to be
consistent
with the normalized activity scores evaluated uing the gradient-based approach. Whereas, the gradient-free
approach yields consistent results for the $A_i$'s, it does not capture the sensitivity associated with
$E_{a,15}$. While this observation bolsters our confidence in the gradient-based approach, it clearly
underscores the shortcoming of the gradient-free approach. Numerical errors are incurred when approximating
the model gradients using the local linear approximation in this case. In~\ref{sub:verify}, we assess the
suitability of the gradient-free approach in quantifying the uncertainty in the ignition delay. 
Secondly, our findings indicate that the  
the ignition delay is predominantly sensitivity towards $A_1$, $A_9$, $E_{a,1}$, and $E_{a,15}$ and 
moderately sensitivity towards $A_{15}$ and $A_{17}$. Sensitivity towards the remaining uncertain inputs is
found to be low or negligible. Note that his observation can also be exploited for reducing the 
dimensionality of the problem from 33 inputs to 4-6 inputs depending upon the required level of accuracy.
The sensitivity-driven dimension reduction for uncertainty quantification has been
discussed in our earlier effort~\cite{Vohra:2018}. In this work, however, we focus on dimension reduction using the active 
subspace approach which is shown to yield even greater scope for dimension reduction i.e. from 33 to 1. Furthermore, we have
shown that the dominant eigenspace can be used to approximate global sensitivity measures with no additional effort.
The accuracy of the 1-dimensional surrogate, $\tilde{G}$, obtained using the two approaches is assessed in the following section. 

\subsection{Surrogate Verification}
\label{sub:verify}

The 1-dimensional surrogate ($\tilde{G}$) is investigated for its accuracy as well as the ability to capture the 
uncertainty in the
model output in two ways. Firstly, we estimate the relative L-2 norm of the discrepancy~($\varepsilon_d$)
between estimates of ignition delay in the case of H$_2$/O$_2$ reaction, obtained using the 
model output and the surrogate in the following equation:
%
\be
\varepsilon_d = \frac{\|G(\bm{\xi}) - \tilde{G}(\bm{\xi})\|_2}{\|G(\bm{\xi})\|_2}
\ee
%
The relative norm, $\varepsilon_d$ was estimated to be 8.50$\times10^{-2}$ based on a cross-validation test data
comprising 10$^{4}$ model evaluations. Hence, in the norm-sense, it can be said that the 1-dimensional surrogate is 
reasonably accurate. 

Secondly, we verify the accuracy of the surrogate in a probabilistic setting. In particular, we compare a histogram
plot of the model evaluations (in the cross-validation set) with a PDF evaluated on the same set of samples
using the surrogate, $\tilde{G}$ in Figure~\ref{fig:pdf_33D}. The PDF is evaluated using surrogates constructed for
a 1-dimensional as well as a 2-dimensional active subspace. 
%
\begin{figure}[htbp]
 \begin{center}
  \includegraphics[width=0.45\textwidth]{./Figures/pdf_comp_id_1e4}
\caption{A comparison of the histogram plot based on model evaluations and the PDF based on surrogate
predictions for ignition delay. The two plots are based on samples in the validation set.}
\label{fig:pdf_33D}
\end{center}
\end{figure}
%
In the above comparison, it is observed that the two PDFs are almost identical, thereby confirming a 1-dimensional
active subspace. The modal estimate from either PDF is observed to be in agreement with that
based on the histogram plot. The mean value based on the set of model evaluations used for verification as well
as the 1-dimensional surrogate was found to be 0.133, whereas, the standard deviations were found to be 0.0198 and 0.0160
respectively. Hence, while the surrogate yields an accurate estimation of the mean, the standard deviation is
under-estimated. This can be understood from the SSP in Figure~\ref{fig:hd} (right) which shows that the model
evaluations do not exactly lie on the linear 1-dimensional surrogate used to capture them. Therefore, the
1-dimensional surrogate can be used to estimate the mean as well as the modal estimate of the ignition delay in
the presence of uncertainty in the 33 parameter considered in this application. The uncertainty associated with the
ignition delay though reasonably close to the true value, is under-estimated.
 

%\subsection{Surrogate-induced risk analysis}






























