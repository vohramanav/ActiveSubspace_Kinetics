\section{Active subspaces and sensitivity analysis}
\label{sec:assa}
In this section, we provide key elements of the active subspace methodology in~\ref{sub:ac}, and
provide theoretical connections between the global sensitivity measures: DGSMs, activity scores,
and the total Sobol index in~\ref{sub:gsa}.

\subsection{Active subspaces}
\label{sub:ac}
%We recall some background regarding active subspaces~\cite{Constantine:2015}.
As mentioned earlier, the active subspace is low-dimensional linear subspace that constitutes
important directions in a model's input space~\cite{Constantine:2015}. The effective variability
in the model output due to uncertain inputs is predominantly captured along these directions.
Mathematically, these directions are essentially the eigenvectors of a positive semi-definite
matrix, $\mat{C}$, defined as follows:
%
\be
\mat{C} = \int_\Omega (\nabla_{\bm{\xi}}f)(\nabla_{\bm{\xi}}f)^\top dP_{\bm{\xi}}
\label{eq:C}
\ee
%
where $\Omega$:~$[-1,1]^{N_p}$ is the domain of $\bm{\xi}$, and $N_p$ is the number of uncertain
inputs. $dP_{\bm{\xi}}$ = $p(\bm{\xi})d\bm{\xi}$, where $p(\bm{\xi})$ is the joint probability
distribution of $\bm{\xi}$. Note that the function, $f$ is considered to be continuous and differentiable
in the considered parameter domain. Hence, it is possible that a given $f(\bm{\xi})$ might not admit an
active subspace. However, it is of remarkable interest to investigate if one exists since the subspace could
be exploited to reduce the dimensionality of the problem and hence the associated computational effort.
$\nabla_{\bm{\xi}}f$ denotes a vector with individual components
being partial derivatives of $f$ with respect to the $i^\text{th}$ input, $\xi_i$. Since $\mat{C}$ is symmetric and
positive semi-definite, it admits a real eigenvalue decomposition:
%
\be
\mat{C} = \bm{W}\bm{\Lambda}\bm{W}^\top
\ee
%
where $\bm{\Lambda}$ = diag($\lambda_1,\ldots,\lambda_{N_p}$) with $\lambda_i$'s
 sorted in descending order, $i:\{1,2,\ldots,N_p\}$. The eigenpairs are partitioned about
 the $i^{\text{th}}$ eigenvalue such that the ratio, $\left(\frac{\lambda_i}{\lambda_{i+1}}\right)$ $\gg$ 1
 as follows:
 %
 \be
 \bm{W} = [\bm{W}_1~\bm{W}_2],~~\bm{\Lambda} = \begin{bmatrix}\bm{\Lambda}_1 & \\  &
  \bm{\Lambda}_2 \\
\end{bmatrix}
 \ee
 %
 where $\bm{W}_1$ comprises the dominant eigenspace of $\mat{C}$ regarded as the active subspace,
 and $\bm{\Lambda}_1$ is the corresponding set of eigenvalues. The active subspace is used to 
 construct the so-called active variables, $\bm{w}_1^\top \bm{\xi}$ and $f(\bm{\xi}(\theta))$ is transformed
 as $G(\bm{w}_1^\top \bm{\xi})$. In practice, the set of samples, $\bm{\xi}_i$'s are projected to the
 physical space as $\bm{\theta}_i$'s at which the model output, $f(\bm{\xi}(\theta))$ is computed. 
 The matrix, $\mat{C}$ is typically approximated using pseudo-random sampling techniques such 
 as Monte Carlo and Latin hypercube sampling. The integral in~\eqref{eq:C} is replaced with a
 summation as follows:
 %
 \be
 \mat{C}\approx \hat{\mat{C}} = \frac{1}{N}\sum\limits_{i=1}^{N} 
 (\nabla_{\bm{\xi}}f(\bm{\xi}_i))(\nabla_{\bm{\xi}}f(\bm{\xi}_i))^\top
 = \hat{\bm{W}}\hat{\bm{\Lambda}}\hat{\bm{W}}^\top
 \ee
 %
 It has been shown that the accuracy of approximated dominant eigenspace, $\hat{\bm{W}}_1$ is inversely
 proportional to the difference between the smallest eigenvalue in $\hat{\bm{\Lambda}}_1$ and the
 largest eigenvalue in $\hat{\bm{\Lambda}}_2$~\cite{Constantine:2014}. 
 
 Components of the eigenvectors
 in the active subspace could be used for estimating the so-called activity scores as a measure for
 global sensitivity and also be used for approximating the DGSMs as discussed in the following section.
  
\subsection{Global sensitivity measures}
\label{sub:gsa}
The total Sobol' index ($T_i(f)$) of a model output, $f(\theta_i)$ quantifies the total contribution of the
input, $\theta_i$ to the variance of the output. Mathematically, this can be expressed as follows:
%
\be
T_i(f) = 1 - 
\frac{\V[\mathbb{E}(f|\bm{\theta}_{\sim i})]}{\V(f)},
\label{eq:total}
\ee
%
where $\bm{\theta}_{\sim i}$ is a vector of all but $i^\text{th}$ uncertain input, and $\V$ denotes the variance.
Determining converged estimates of $T_i(f)$ typically involves tens of thousands of model runs and is therefore
prohibitive in the case of compute-intensive applications. The DGSMs~\cite{Sobol:2009}
provide a means for approximating the upper bound on $T_i(f)$ with fewer computations as 
discussed in~\cite{Vohra:2018}. 
\smallskip

For $f: \Omega \to \R$, we consider the DGSMs,
\[
    \nu_i(f) := \E{\left(\frac{\partial f}{\partial\xi_i}\right)^2} =
                  \int_\Omega 
                  \left(\frac{\partial f}{\partial\xi_i}\right)^2
                  \pi(\vec{\xi})d\vec{\xi}, \quad i = 1, \ldots, \Np.   
\]
Here $\pi$ is the joint PDF of $\xi$. 
Let $\mat{C} = \E{\nabla f \nabla f^\top}$ and note that
\[
   \nu_i(f) = \vec{e}_i^\top \mat{C} \vec{e}_i, 
\]
where $\vec{e}_i$ is the $i$th coordinate vector in $\R^\Np$.
This also shows that $\sum_{j=1}^\Np \nu_i(f) = \trace(\mat{C})$.
We consider the spectral decomposition of $\mat{C}$, given by  
$\mat{C} = \sum_{k=1}^\Np \lambda_k \vec{u}_k \vec{u}_k^\top$, where 
$\lambda_i$ are the (non-negative) eigenvalues of $\mat{C}$ and
$\vec{u}_k$ are the corresponding (orthonormal) eigenvectors.
We note that,
\begin{equation}\label{equ:spectral_DGSM}
\nu_i(f) = \vec{e}_i^T \Big(\sum_{k=1}^\Np \lambda_k \vec{u}_k \vec{u}_k^\top\Big) \vec{e}_i
 = \sum_{k=1}^\Np \lambda_k \ip{\vec{e}_i}{\vec{u}_k}^2, 
\end{equation}
which gives a spectral representation for the DGSMs. Truncating the summation 
leads to the notion of activity scores introduced in~\cite{Diaz:2016,Constantine:2017}
defined as follows: 
\[
   \alpha_i(f; r) =  \sum_{k=1}^r \lambda_k \ip{\vec{e}_i}{\vec{u}_k}^2,
   \quad j = 1, \ldots, \Np, \quad r \leq \Np.
\]
The activity scores connect ideas from active subspaces and global sensitivity
analysis, and can be used to approximate DGSMs.  The following result, which
can be found in~\cite{Diaz:2016,Constantine:2017}, quantifies the error in this
approximation. We provide a short proof for completeness. 
\begin{proposition}\label{prp:dgsm_bound} 
For $1 \leq r \leq \Np$,
\[
0 \leq \nu_i(f) - \alpha_i(f; r) \leq \lambda_{r+1}, \quad i = 1, \ldots, \Np.
\] 
\end{proposition}
\begin{proof} 

%Using the spectral representation of the DGSMs and the definition of activity
%scores we clearly see:
%\[
%\alpha_i(f;r) = \sum_{k=1}^{r}\lambda_k \langle \vec{e}_i, \vec{u}_k \rangle^2 \leq \sum_{k=1}^{N_p}\lambda_k \langle \vec{e}_i, \vec{u}_k \rangle^2 = \nu_i(f), \quad \quad i = 1,\ldots,N_p, \quad r \leq N_p
%\]
%In other words
%\[
%0 \leq  \sum_{k=1}^{N_p}\lambda_k \langle \vec{e}_i, \vec{u}_k \rangle^2 - \sum_{k=1}^{r}\lambda_k \langle \vec{e}_i, \vec{u}_k \rangle^2,  \quad \quad i = 1,\ldots,N_p
%\]
%with equality if $N_p=r$.
%\newline
%We can write:
%\[
%\begin{aligned}
%\nu_i(f) = \sum_{k=1}^{N_p}\lambda_k \langle \vec{e}_i, \vec{u}_k \rangle^2 = \sum_{k=1}^{r}\lambda_k \langle \vec{e}_i, \vec{u}_k \rangle^2 + \sum_{k=r+1}^{N_p}\lambda_k \langle \vec{e}_i, \vec{u}_k \rangle^2  \\
%= \alpha_i(f;r) + \sum_{k=r+1}^{N_p}\lambda_k \langle \vec{e}_i, \vec{u}_k \rangle^2 \leq \alpha_i(f;r) + \lambda_{r+1} \sum_{k=r+1}^{N_p} \langle \vec{e}_i, \vec{u}_k \rangle^2,  \quad \quad i = 1,\ldots,N_p
%\end{aligned}
%\]
%The eigenvectors $\vec{u}_k$ are orthonormal so they all have length 1. Also note that for every $x \in \R^n$ we have \[ \norm{x}^2 = \sum_{k=1}^{n} \langle \vec{x}, \vec{u}_k \rangle^2\] This is known as Parseval's identity.
%In particular in this case $\vec{x} = \vec{e}_i \in \R^{N_p}$ so
%\[1 = \norm{\vec{e}_i}^2 = \sum_{k=r+1}^{N_p} \langle \vec{e}_i, \vec{u}_k \rangle^2\]
%Finally we write:
%\[
%\nu_i(f) \leq \alpha_i(f;r) + \lambda_{r+1}, \quad \quad i = 1,\ldots,N_p
%\]

Note that, $\nu_i(f) - \alpha(f; r)  = \sum_{k=r+1}^\Np \lambda_k \ip{\vec{e}_i}{\vec{u}_k}^2 \geq 0$,
which gives the first inequality. To see the upper bound, we note,
\[
   \sum_{k=r+1}^\Np \lambda_k \ip{\vec{e}_i}{\vec{u}_k}^2 \leq \lambda_{r+1} \sum_{k=r+1}^\Np \ip{\vec{e}_i}{\vec{u}_k}^2
   \leq \lambda_{r+1}. 
\]
The last inequality holds because 
$1 = \norm{\vec{e}_i}^2 = 
\sum_{k = 1}^\Np \ip{\vec{e}_i}{\vec{u}_k}^2 
\geq \sum_{k=r+1}^\Np \ip{\vec{e}_i}{\vec{u}_k}^2$.
\end{proof} 
The utility of this result is realized in problems with 
high-dimensional parameter spaces in which 
the eigenvalues $\lambda_i$ decay rapidly to zero; in 
such cases, this result implies that  $\nu_i(f) \approx \alpha_i(f; r)$,
where $r$ is the \emph{numerical rank} of $\mat{C}$.  This will be especially
effective if there is a large gap in the eigenvalues.  

Consider a function $y = f(\xi_1, \xi_2, \ldots, \xi_\Np)$ where
$\xi_i$ are independent are either uniformly distributed or
distributed according to a Boltzman probability distribution.  
It was shown in~\cite{Lamboni:2013} that the total Sobol' 
index $T_i(f)$ can be bounded in terms of $\nu_i(f)$:
\begin{equation}\label{equ:sobol_bound}
T_i(f) \leq \frac{C_i}{\V(f)}\nu_i(f), \quad i = 1, \ldots, \Np,
\end{equation}
where for each $i$, $C_i$ is an appropriate \emph{Poincar\'{e}} constant.
This provides a strong theoretical basis for using DGSMs to identify 
unimportant inputs. Moreover, 
in~\cite{Vohra:2018} we introduced the normalized screening metric
\[
   \tilde{\nu}_i(f) = \frac{C_i \nu_i(f)}{\sum_{i=1}^\Np C_i \nu_i(f)},
\]
for detecting unimportant input parameters. 
Note that if $\xi_i$ are iid, the $C_i$'s will cancel and the normalized screening
metric reduces to 
\[
    \tilde{\nu}_i(f) = \frac{\nu_i(f)}{\sum_{i=1}^\Np \nu_i(f)} 
      = \frac{\vec{e}_i^T \mat{C} \vec{e}_i}{\trace(\mat{C})} 
      = \frac{\sum_{k=1}^\Np \lambda_k \ip{\vec{e}_i}{\vec{u}_k}^2}{\sum_{k = 1}^\Np \lambda_k}.
\]
The following result is a straightforward generalization of
the result from~\cite{Diaz:2016}, and 
links activity scores and Sobol' indices.
\begin{proposition}
For $1 \leq r \leq \Np$, and $i = 1, \ldots, \Np$, we have
\[
       T_i(f) \leq \frac{C_i}{\V(f)}(\alpha_i(f; r) + \lambda_{r+1}).
\]
\end{proposition}
\begin{proof}
The result follows immediately from Proposition~\ref{prp:dgsm_bound} and~\eqref{equ:sobol_bound}.
\end{proof}
We can also bound the normalized DGSMs using activity scores; it is easy to see  
\[
\tilde{\nu}_i(f) \leq 
\frac{ C_i \big(\alpha_i(f; r) + \lambda_{r+1}\big)}{\sum_{i=1}^\Np C_i \alpha_i(f; r)}
=\frac{C_i \alpha_i(f; r)}{\sum_{i=1}^\Np C_i \alpha_i(f; r)} + \kappa_i \lambda_{r+1}, 
\]
whith $\kappa_i = C_i / (\sum_i C_i \alpha_i(f; r))$. 
In the case where where $\lambda_{r+1} \approx 0$, 
this motivates definition of
normalized activity scores
\[
   \tilde{\alpha}_i(f; r) =  \frac{C_i \alpha_i(f; r)}{\sum_{i=1}^\Np C_i \alpha_i(f; r)}.
\] 

