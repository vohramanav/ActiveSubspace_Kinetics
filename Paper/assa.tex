\newcommand{\act}[3]{\nu_{{#2},{#3}}({#1})}
\newcommand{\actt}[3]{\tilde{\nu}_{{#2},{#3}}({#1})}
\newcommand{\redf}[2]{f^{({#1})}(\vec\xi; {#2})}
\section{Active subspaces}
\label{sub:ac}

Herein, we use a random vector 
$\vec\xi \in \Omega\in\mathbb{R}^{N_p}$ to parameterize model uncertainties, 
where $N_p$ is the number of uncertain inputs.
In practical computations, the \emph{canonical} variables $\xi_i, i=1,\ldots ,N_p$, are  
mapped to physical ranges meaningful in a given mathematical model. 
As mentioned in the introduction, an active subspace is a low-dimensional subspace
that consists of important directions in a model's input
parameter space~\cite{Constantine:2015}. The effective variability in a model output $f$
due to uncertain inputs is predominantly captured
along these directions. 
The directions constituting the active subspace are the dominant eigenvectors of the positive
semidefinite matrix 
%
\be
\mat{C} = \int_\Omega (\nabla_{\vec{\xi}}f)(\nabla_{\vec{\xi}}f)^\top \mu(d\vec\xi), 
\label{eq:C}
\ee
%
with 
$\mu(d\vec{\xi}) = \pi(\vec{\xi})d\vec{\xi}$, where $\pi(\vec{\xi})$ is the joint probability
distribution of $\vec{\xi}$. Herein, $f$ is assumed to be a square integrable 
function with continuous partial 
derivatives with respect to the input parameters; moreover, we assume the partial derivatives
are square integrable. 
%Hence, it is possible that a given $f(\vec{\xi})$ might not admit an active
%subspace. However, it is of remarkable interest to investigate if one exists
%since the subspace could be exploited to reduce the dimensionality of the
%problem and hence the associated computational effort.
%Here,
%$\nabla_{\vec{\xi}}f$ denotes the gradient vector. 
%with individual components
%being partial derivatives of $f$ with respect to the $i^\text{th}$ input, $\xi_i$. 
Since $\mat{C}$ is symmetric and
positive semidefinite, it admits a spectral decomposition:
%
\be
\mat{C} = \mat{W}\mat{\Lambda}\mat{W}^\top.
\ee
%
Here $\mat{\Lambda}$ = diag($\lambda_1,\ldots,\lambda_{N_p}$) with the eigenvalues
$\lambda_i$'s sorted in descending order
\[
     \lambda_1 \geq \lambda_2 \geq \cdots \geq \lambda_\Np \geq 0,
\] 
and $\mat{W}$ has the (orthonormal) eigenvectors $\vec{w}_1, \ldots, \vec{w}_\Np$ as its columns.
The eigenpairs are partitioned about the $r$th eigenvalue such that
$\lambda_r/\lambda_{r+1}\gg 1$, 
\be
 \mat{W} = [\mat{W}_1~\mat{W}_2],~~\mat{\Lambda} = \begin{bmatrix}\mat{\Lambda}_1 & \\  &
  \mat{\Lambda}_2. 
\end{bmatrix}
\ee
 %
The columns of $\mat{W}_1 = 
\begin{bmatrix} \vec{w}_r \cdots \vec{w}_r\end{bmatrix}$ 
span the dominant eigenspace of $\mat{C}$ and
define the active subspace, and $\mat{\Lambda}_1$ is a diagonal matrix with
the corresponding set of eigenvalues, $\lambda_1, \ldots, \lambda_r$, on its diagonal. 
Once the active subspace
is computed, dimension reduction is accomplished by transforming the parameter
vector $\vec\xi$ into 
$\vec{y} = \mat{W}_1^\top\vec{\xi} \in \R^r$. The elements of $\vec{y}$ are 
referred to as the set of active variables. 
%The number $r$ of active
%variables is equal to the number of dominant eigenvectors. 

Consider the function
\[
    G(\vec{y}) = f(\mat{W}_1\vec{y}), \quad \vec{y} \in \R^r.
\]
Following~\cite{Constantine:2015}, we use the approximation 
\[
f(\vec{\xi}) \approx f(\mat{W}_1 \mat{W}_1^T \vec\xi) =  
G(\mat{W}_1^\top \vec{\xi}).
\] 
That is, the model output $f(\vec\xi)$, in the original parameter space,
is approximated by $G(\mat{W}_1^\top \vec{\xi})$ in the active subspace.
%Hence, $f$ is essentially evaluated at the projection of $\vec{\xi}$ on the
%column space of $\mat{W}_1$. 
We could confine uncertainty analysis to the inputs in the
active subspace whose dimension is typically much smaller (in applications that
admit such a subspace) than the dimension of the original input parameter. To further
expedite uncertainty analysis, one could fit a regression surface to $G$ using the 
following sequence of steps, as outlined in~\cite[chapter 4]{Constantine:2015}. 
\begin{enumerate}
\item Consider a given set of $N$ 
data points, $\big(\vec{\xi}_i, f(\vec{\xi}_i)\big)$, $i = 1, \ldots, N$. 
\item For each $\vec{\xi}_i$, compute $\vec{y}_i = \mat{W}_1^\top\vec{\xi}_i$. Note that
 $G(\vec{y}_i)$ $\approx$ $f(\vec{\xi}_i)$.
\item Use data points $\big(\vec{y}_i, f(\vec\xi_i)\big)$, $i = 1, \ldots, N$, to compute a 
regression surface $\tilde{G}(\vec{y})\approx 
G(\vec{y})$.
\item Overall approximation, $f(\vec{\xi})$ $\approx$ $\tilde{G}(\mat{W}_1^\top\vec{\xi})$.
\end{enumerate}

In practice, the matrix $\mat{C}$ defined in~\eqref{eq:C} is 
approximated using pseudo-random sampling techniques such as Monte Carlo or
Latin hypercube sampling:
%The integral in~\eqref{eq:C} is replaced with a
%summation as follows:
 %
 \be
 \mat{C}\approx \hat{\mat{C}} = \frac{1}{N}\sum\limits_{i=1}^{N} 
 (\nabla_{\vec{\xi}}f(\vec{\xi}_i))(\nabla_{\vec{\xi}}f(\vec{\xi}_i))^\top
 = \hat{\mat{W}}\hat{\mat{\Lambda}}\hat{\mat{W}}^\top
\label{eq:chat}
 \ee
 %
Clearly the computational effort associated with constructing the matrix
$\hat{\mat{C}}$ scales with the number of samples, $N$. Hence, an iterative
computational approach is adopted in this work to gradually increase  
$N$ until the dominant eigenpairs are approximated
with sufficient accuracy; see Section~\ref{sec:method}. 

%%It has been shown that the accuracy of approximated dominant eigenspace,
%%$\hat{\vec{w}}_1$ is inversely proportional to the difference between the
%%smallest eigenvalue in $\hat{\vec{\Lambda}}_1$ and the largest eigenvalue in
%%$\hat{\vec{\Lambda}}_2$~\cite{Constantine:2014}. 
%%Components of the eigenvectors in the active subspace could be used for
%%estimating the so-called activity scores as a measure for global sensitivity
%%and also be used for approximating the DGSMs as discussed in the following
%%section.
  

\section{GSA measures and their links with active subspaces}
\label{sub:gsa}
Consider a function $f = f(\xi_1, \xi_2, \ldots, \xi_\Np)$. 
While the active subspace framework described above does not make any assumptions
about statistical independence of the inputs $\xi_i$, $i = 1, \ldots, \Np$, 
the classical 
framework of variance based sensitivity analysis~\cite{Sobol:2001, Saltelli:2010} 
assumes that the inputs
are statistically independent. While extensions to the cases 
of correlated inputs exist~\cite{Borgonovo:2007,Li:2010,Jacques:2006,Xu:2007,Hart:2017},
 we limit the discussion in this section to the
case of random inputs that are statistically independent and are 
either uniformly distributed or
distributed according to the Boltzmann probability distribution.
Note that a measure $\mu$
on $\R$ is referred to as a Boltzmann measure if it is 
absolutely continuous with respect to the Lebesgue measure  
and admits a density  of the form $\pi(x) = C \exp\{-V(x)\}$,
where $V$ is a continuous function and $C$ a normalization 
constant~\cite{Lamboni:2013}.


The total Sobol' index ($T_i(f)$) of a model output, $f(\vec\xi)$ quantifies
the total contribution of the input, $\xi_i$ to the variance of the
output~\cite{Sobol:2001}. Mathematically, this can be expressed as follows:
%
\be
T_i(f) = 1 - 
\frac{\V[\mathbb{E}(f|\vec{\xi}_{\sim i})]}{\V(f)},
\label{eq:total}
\ee
%
where $\vec{\xi}_{\sim i}$ is the input parameter vector with the  
$i^\text{th}$ entry removed, and $\V$ denotes the variance. The total Sobol' index accounts
for the contribution of a given input to the variability in the output by itself
as well as due to its interaction or coupling with other inputs. 
Determining accurate estimates of $T_i(f)$ typically involves a large 
number of 
model runs and is therefore can be prohibitive in the case of
compute-intensive applications. Derivative based 
global sensitivity measures (DGSMs)~\cite{Sobol:2009} provide a means for
approximating informative upper bounds on $T_i(f)$ at a lower cost; see 
also~\cite{Vohra:2018}. 

For $f: \Omega \to \R$, we consider the DGSMs,
\[
    \nu_i(f) := \E{\left(\frac{\partial f}{\partial\xi_i}\right)^2} =
                  \int_\Omega 
                  \left(\frac{\partial f}{\partial\xi_i}\right)^2
                  \pi(\vec{\xi})d\vec{\xi}, \quad i = 1, \ldots, \Np.   
\]
Here $\pi$ is the joint PDF of $\vec\xi$. 
Let $\mat{C}$ be as in~\eqref{eq:C}, and consider the spectral decomposition written 
as $\mat{C} = \sum_{k=1}^\Np \lambda_k \vec{w}_k \vec{w}_k^\top$. Herein, we use the notation 
$\ip{\cdot}{\cdot}$ for the Euclidean inner product.
% where
%$\lambda_i$ are the (non-negative) eigenvalues of $\mat{C}$, in descending
%order, and
%$\vec{w}_k$ are the corresponding (orthonormal) eigenvectors. 
The following
result provides a representation of DGSMs in terms of the 
spectral representation of $\mat{C}$: 
\begin{lemma}
We have
$\nu_i(f) = \sum_{k=1}^\Np \lambda_k \ip{\vec{e}_i}{\vec{w}_k}^2$.
\end{lemma}
\begin{proof}
Note that $\nu_i(f) = \vec{e}_i^\top \mat{C} \vec{e}_i$,  
where $\vec{e}_i$ is the $i$th coordinate vector in $\R^\Np$, $i = 1, \ldots, \Np$.
Therefore,
$\nu_i(f) = \vec{e}_i^T \Big(\sum_{k=1}^\Np \lambda_k \vec{w}_k \vec{w}_k^\top\Big) \vec{e}_i
 = \sum_{k=1}^\Np \lambda_k \ip{\vec{e}_i}{\vec{w}_k}^2$. 
\end{proof}
In the case where the eigenvalues decay rapidly to zero, we can obtain
accurate approximations of $\nu_i(f)$ by truncating the summation: 
\[
   \act{f}{i}{r} =  \sum_{k=1}^r \lambda_k \ip{\vec{e}_i}{\vec{w}_k}^2,
   \quad i = 1, \ldots, \Np, \quad r \leq \Np.
\]
The quantities $\act{f}{i}{r}$ are called activity scores
in~\cite{Diaz:2016,Constantine:2017}, where links between GSA measures and
active subspaces is explored.
%The activity scores connect ideas from active subspaces and global sensitivity
%analysis, and can be used to approximate DGSMs.  
The following result, which
can also be found in~\cite{Diaz:2016,Constantine:2017}, quantifies the error in this
approximation. We provide a short proof for completeness. 
\begin{proposition}\label{prp:dgsm_bound} 
For $1 \leq r < \Np$,
\[
0 \leq \nu_i(f) - \act{f}{i}{r} \leq \lambda_{r+1}, \quad i = 1, \ldots, \Np.
\] 
\end{proposition}
\begin{proof} 

%Using the spectral representation of the DGSMs and the definition of activity
%scores we clearly see:
%\[
%\alpha_i(f;r) = \sum_{k=1}^{r}\lambda_k \langle \vec{e}_i, \vec{w}_k \rangle^2 \leq \sum_{k=1}^{N_p}\lambda_k \langle \vec{e}_i, \vec{w}_k \rangle^2 = \nu_i(f), \quad \quad i = 1,\ldots,N_p, \quad r \leq N_p
%\]
%In other words
%\[
%0 \leq  \sum_{k=1}^{N_p}\lambda_k \langle \vec{e}_i, \vec{w}_k \rangle^2 - \sum_{k=1}^{r}\lambda_k \langle \vec{e}_i, \vec{w}_k \rangle^2,  \quad \quad i = 1,\ldots,N_p
%\]
%with equality if $N_p=r$.
%\newline
%We can write:
%\[
%\begin{aligned}
%\nu_i(f) = \sum_{k=1}^{N_p}\lambda_k \langle \vec{e}_i, \vec{w}_k \rangle^2 = \sum_{k=1}^{r}\lambda_k \langle \vec{e}_i, \vec{w}_k \rangle^2 + \sum_{k=r+1}^{N_p}\lambda_k \langle \vec{e}_i, \vec{w}_k \rangle^2  \\
%= \alpha_i(f;r) + \sum_{k=r+1}^{N_p}\lambda_k \langle \vec{e}_i, \vec{w}_k \rangle^2 \leq \alpha_i(f;r) + \lambda_{r+1} \sum_{k=r+1}^{N_p} \langle \vec{e}_i, \vec{w}_k \rangle^2,  \quad \quad i = 1,\ldots,N_p
%\end{aligned}
%\]
%The eigenvectors $\vec{w}_k$ are orthonormal so they all have length 1. Also note that for every $x \in \R^n$ we have \[ \norm{x}^2 = \sum_{k=1}^{n} \langle \vec{x}, \vec{w}_k \rangle^2\] This is known as Parseval's identity.
%In particular in this case $\vec{x} = \vec{e}_i \in \R^{N_p}$ so
%\[1 = \norm{\vec{e}_i}^2 = \sum_{k=r+1}^{N_p} \langle \vec{e}_i, \vec{w}_k \rangle^2\]
%Finally we write:
%\[
%\nu_i(f) \leq \alpha_i(f;r) + \lambda_{r+1}, \quad \quad i = 1,\ldots,N_p
%\]

Note that, $\nu_i(f) - \act{f}{i}{r}= \sum_{k=r+1}^\Np \lambda_k \ip{\vec{e}_i}{\vec{w}_k}^2 \geq 0$,
which gives the first inequality. To see the upper bound, we note,
\[
   \sum_{k=r+1}^\Np \lambda_k \ip{\vec{e}_i}{\vec{w}_k}^2 \leq \lambda_{r+1} \sum_{k=r+1}^\Np \ip{\vec{e}_i}{\vec{w}_k}^2
   \leq \lambda_{r+1}. 
\]
The last inequality holds because 
$1 = \|\vec{e}_i\|_2^2 = 
\sum_{k = 1}^\Np \ip{\vec{e}_i}{\vec{w}_k}^2 
\geq \sum_{k=r+1}^\Np \ip{\vec{e}_i}{\vec{w}_k}^2$.
\end{proof} 
The utility of this result is realized in problems with 
high-dimensional parameters in which 
the eigenvalues $\lambda_i, i=1,\ldots,N_p$, decay rapidly to zero; in 
such cases, this result implies that  $\nu_i(f) \approx \act{f}{i}{r}$,
where $r$ is the \emph{numerical rank} of $\mat{C}$.  This will be especially
effective if there is a large gap in the eigenvalues.  

The relations recored in the following lemma will be useful in the discussion 
that follows.
\begin{lemma}\label{lem:sum}
We have
\begin{enumerate}[label=(\alph*)]
\item $\sum_{i = 1}^\Np \act{f}{i}{r} = \sum_{k = 1}^r \lambda_k$. 
\item $\sum_{i = 1}^\Np \nu_i(f) = \sum_{k = 1}^\Np \lambda_k$. 
\end{enumerate}
\end{lemma}
\begin{proof}
The first statement of the lemma holds, because
\[
\sum_{i=1}^\Np \act{f}{i}{r}    
= \sum_{i=1}^\Np \sum_{k=1}^r \lambda_k \ip{\vec{e}_i}{\vec{w}_k}^2 
= \sum_{k=1}^r \lambda_k \sum_{i=1}^\Np \ip{\vec{e}_i}{\vec{w}_k}^2 
= \sum_{k=1}^r \lambda_k \| \vec{w}_k \|^2 = \sum_{k=1}^r \lambda_k.
\]
The statement (b) follows immediately from (a), because $\nu_i(f) = \act{f}{i}{\Np}$.
%$\sum_{i=1}^\Np \nu_i(f) = \trace(\mat{C}) = \sum_{k = 1}^\Np \lambda_k$.
\end{proof}

It was shown in~\cite{Lamboni:2013} that the total Sobol' 
index $T_i(f)$ can be bounded in terms of $\nu_i(f)$:
\begin{equation}\label{equ:sobol_bound}
T_i(f) \leq \frac{C_i}{\V(f)}\nu_i(f), \quad i = 1, \ldots, \Np,
\end{equation}
where for each $i$, $C_i$ is an appropriate \emph{Poincar\'{e}} constant
that depends on the distribution of $\xi_i$.
For instance, if $\xi_i$ is uniformly distributed on $[-1, 1]$, then $C_i = 4/\pi^2$; and in the 
case $\xi_i$ is normally distributed with variance $\sigma_i^2$, then $C_i = \sigma_i^2$. 
Note that~\eqref{equ:sobol_bound} for the special cases of 
uniformly distributed or normally distributed inputs was established first in~\cite{Sobol:2009}.
The bound~\eqref{equ:sobol_bound} provides a strong theoretical basis for using DGSMs to identify 
unimportant inputs. 

Combining Proposition~\ref{prp:dgsm_bound} and~\eqref{equ:sobol_bound}, shows
an interesting link between the activity scores and total Sobol' indices.
Specifically, by computing the activity scores, we can identify the unimportant
inputs.  
Subsequently, 
one can attempt to reduce parameter dimension by fixing
unimportant inputs at nominal values. 

Suppose activity scores
are used to approximate DGSMs, and suppose
$\xi_i$ is
deemed unimportant as a result, due to a small activity score. 
We want to estimate
the approximation error that occurs once $\xi_i$ is fixed at a nominal value.
To formalize this process, we proceed as follows.
Let $\vec\xi$ be given and let $z$ be a nominal value for $\xi_i$.  
%we define 
%$\vec{y}^z(\vec\xi)$ as the vector with entries $y^z_j = \xi_j$ for $j \neq i$
%and $y^z_i = z$.
Consider the \emph{reduced} model, 
obtained by fixing $\xi_i$ at the nominal value: 
\[
\redf{i}{z} = f(\xi_1, \xi_2, \ldots, \xi_{i-1}, z, \xi_{i+1}, \ldots, \xi_\Np),
\] 
and consider the following relative error indicator:
\[
\mathcal{E}(z) =
\frac{ \int_\Omega \big( f(\vec\xi) - \redf{i}{z}\big)^2 \, \mu(d\vec\xi) }
          {\int_\Omega f(\vec\xi)^2 \, \mu(d\vec\xi)}.
\] 
This error indicator is a function of $z$ with $z$ distributed 
according to the distribution of $\xi_i$.
\begin{theorem}\label{thm:error_estimate}
We have $\Ez{ \mathcal{E}(z)} \leq 2C_i\big(\act{f}{i}{r} + \lambda_{r+1}\big)/{\V(f)}$, 
for $1 \leq r < \Np$.
\end{theorem}
\begin{proof} 
Note that, since 
$\int_\Omega f(\vec\xi)^2 \, \mu(d\vec\xi) = \V(f) + 
\left(\int_\Omega f(\vec\xi) \, \mu(d\vec\xi)\right)^2 \geq \V(f)$, we have
\[
\Ez{ \mathcal{E}(z)} \leq \frac{1}{\V(f)} \Ez{ 
\int_\Omega \big( f(\vec\xi) - \redf{i}{z}\big)^2 \, \mu(d\vec\xi)}
= 2 T_i(f), 
\]
where the equality can be shown using arguments similar to the proof of the main result 
in~\cite{SobolTarantolaGatelliEtAl07}. Using this, along with~\eqref{equ:sobol_bound} and
Proposition~\ref{prp:dgsm_bound}, we have 
\[
\Ez{ \mathcal{E}(z)} \leq 
\frac{2C_i}{\V(f)}\nu_i(f)
\leq 
\frac{2C_i}{\V(f)}\big[\act{f}{i}{r} + \lambda_{r+1}\big]. \qedhere
\]
\end{proof}

In~\cite{Vohra:2018} the screening metric
\be
   \tilde{\nu}_i(f) = \frac{C_i \nu_i(f)}{\sum_{i=1}^\Np C_i \nu_i(f)},
\label{eq:ndgsm}
\ee
was shown to be useful for detecting unimportant inputs. 
We can also bound the normalized DGSMs using activity scores as follows. It is straightforward to see  
that
\[
\tilde{\nu}_i(f) \leq 
\frac{ C_i \big(\act{f}{i}{r} + \lambda_{r+1}\big)}{\sum_{i=1}^\Np C_i \act{f}{i}{r}}
=\frac{C_i \act{f}{i}{r}}{\sum_{i=1}^\Np C_i \act{f}{i}{r}} + \kappa_i \lambda_{r+1}, 
\]
with $\kappa_i = C_i / (\sum_i C_i \act{f}{i}{r})$. 
In the case where where $\lambda_{r+1} \approx 0$, 
this motivates definition of
normalized activity scores
\[
   \actt{f}{i}{r} =  \frac{C_i \act{f}{i}{r}}{\sum_{i=1}^\Np C_i \act{f}{i}{r}}.
\] 

\begin{remark}
If the random inputs $\xi_i$, $i = 1, \ldots, \Np$, are iid, then 
the $C_i$'s in the definition of the normalized screening metric will cancel and 
\[
    \tilde{\nu}_i(f) = \frac{\nu_i(f)}{\sum_{i=1}^\Np \nu_i(f)} 
  %    = \frac{\vec{e}_i^T \mat{C} \vec{e}_i}{\trace(\mat{C})} 
      = \frac{\sum_{k=1}^\Np \lambda_k \ip{\vec{e}_i}{\vec{w}_k}^2}{\sum_{k = 1}^\Np \lambda_k}.
\]
The expression for the denominator follows from Lemma~\ref{lem:sum}(b). 
Also, in the iid case, using Lemma~\ref{lem:sum}(a) we can simplify the normalized activity scores as follows. 
\[
   \actt{f}{i}{r} =  \frac{\act{f}{i}{r}}{\sum_{i=1}^\Np \act{f}{i}{r}} = 
                     \frac{\sum_{k=1}^r \lambda_k \ip{\vec{e}_i}{\vec{w}_k}^2}
                          {\sum_{k = 1}^r \lambda_k}.
\]
\end{remark}

The significance of the developments in this section are as follows.
Theorem~\ref{thm:error_estimate} provides a theoretical basis for parameter dimension
reduction using activity scores. This is done by providing an estimate of the
error between the reduced model and the original model. If a precise ranking of
parameter importance based on total Sobol' indices is desired, one
can first identify unimportant inputs by computing activity scores, and then
perform a detailed variance based GSA of the remaining model parameters. This approach will 
provide great computational savings as variance based GSA will now be performed only
for a small number of
inputs deemed important based on their activity scores. Moreover, the
presented result covers a broad class of input distributions coming from 
Boltzmann family of distributions.
Additionally, the normalized activity scores discussed above provide practical
screening metrics that require only computing the activity scores. This is in
contrast to the bound in Theorem~\ref{thm:error_estimate} that requires the
variance $\V(f)$ of the model output.
