\section{Active subspaces}
\label{sub:ac}

Herein, we use a random vector 
$\vec\xi \in \Omega\in\mathbb{R}^{N_p}$ to parameterize model uncertainties, 
where $N_p$ is the number of uncertain inputs.
In practical computations, the \emph{canonical} variables $\xi_i, i=1,\ldots ,N_p$, are  
mapped to physical ranges meaningful in a given mathematical model. 
As mentioned in the introduction, an active subspace is a low-dimensional subspace
that constitutes important directions in a model's input
parameter space~\cite{Constantine:2015}. The effective variability in a model output $f$
due to uncertain inputs is predominantly captured
along these directions. 
The directions spanning the active subspace are the eigenvectors of the positive
semidefinite matrix 
%
\be
\mat{C} = \int_\Omega (\nabla_{\vec{\xi}}f)(\nabla_{\vec{\xi}}f)^\top \mu(d\vec\xi), 
\label{eq:C}
\ee
%
with 
$\mu(d\vec{\xi}) = \pi(\vec{\xi})d\vec{\xi}$, where $\pi(\vec{\xi})$ is the joint probability
distribution of $\vec{\xi}$. Herein, $f$ is assumed to be a function with continuous partial 
derivatives with respect to the input parameters. 
%Hence, it is possible that a given $f(\vec{\xi})$ might not admit an active
%subspace. However, it is of remarkable interest to investigate if one exists
%since the subspace could be exploited to reduce the dimensionality of the
%problem and hence the associated computational effort.
%Here,
%$\nabla_{\vec{\xi}}f$ denotes the gradient vector. 
%with individual components
%being partial derivatives of $f$ with respect to the $i^\text{th}$ input, $\xi_i$. 
Since $\mat{C}$ is symmetric and
positive semidefinite, it admits a spectral decomposition:
%
\be
\mat{C} = \mat{W}\mat{\Lambda}\mat{W}^\top.
\ee
%
Here $\mat{\Lambda}$ = diag($\lambda_1,\ldots,\lambda_{N_p}$) with the eigenvalues
$\lambda_i$'s sorted in descending order
\[
     \lambda_1 \geq \lambda_2 \geq \cdots \geq \lambda_\Np \geq 0,
\] 
and $\mat{W}$ has the eigenvectors $\vec{w}_1, \ldots, \vec{w}_\Np$ as its columns.
The eigenpairs are partitioned about the $i^{\text{th}}$ eigenvalue such that
the ratio, $\lambda_i/\lambda_{i+1}\gg 1$ as follows:
\be
 \mat{W} = [\mat{W}_1~\mat{W}_2],~~\mat{\Lambda} = \begin{bmatrix}\mat{\Lambda}_1 & \\  &
  \mat{\Lambda}_2. 
\end{bmatrix}
\ee
 %
The columns of $\mat{W}_1$ span the dominant eigenspace of $\mat{C}$ and
defines the active subspace, and $\mat{\Lambda}_1$ is a diagonal matrix with
the corresponding set of eigenvalues on its diagonal. Once the active subspace
is computed, dimension reduction is accomplished by transforming the parameter
vector $\vec\xi$ into 
$\vec{y} = \mat{W}_1^\top\vec{\xi}$. The elements of $\vec{y}$ are 
referred to as the set of active variables. The number of active
variables is equal to the number of dominant eigenvectors. Consider the function
\[
    G(\vec{y}) = f(\mat{W}\vec{y}).
\]
The model output in the original space, $f(\vec{\xi})$,
is approximated by $G(\mat{W}_1^\top \vec{\xi})$ in the active subspace~\cite{Constantine:2015}.
%Hence, $f$ is essentially evaluated at the projection of $\vec{\xi}$ on the
%column space of $\mat{W}_1$. 
We could confine uncertainty analysis to the inputs in the
active subspace whose dimension is typically much smaller (in applications that
admit such a subspace) than the dimension of the original input space. To further
expedite uncertainty analysis, one could fit a regression surface to $G$ using the 
following sequence of steps, as outlined in~\cite[chapter 4]{Constantine:2015}. 
\begin{enumerate}
\item Consider a set of input-output pairs, $\{\vec{\xi}_i, f(\vec{\xi}_i)\}$, $i = 1, \ldots, N$. {\hayleynote {Do we need to define what N is?}}
\item For each $\vec{\xi}_i$, compute $\vec{y}_i = \mat{W}_1^\top\vec{\xi}_i$. Note that
 $G(\mat{W}_1^\top \vec{\xi}_i)$ $\approx$ $f(\vec{\xi}_i)$.
\item Use $\{\vec{y}_i\}_{i =1}^N$ to compute a 
regression surface, $\tilde{G}(\vec{y})\approx 
G(\vec{y})$.
\item Overall approximation, $f(\vec{\xi})$ $\approx$ $\tilde{G}(\mat{W}_1^\top\vec{\xi})$.
\end{enumerate}

In practice, the matrix $\mat{C}$ defined in~\eqref{eq:C} is 
approximated using pseudo-random sampling techniques such as Monte Carlo or
Latin hypercube sampling:
%The integral in~\eqref{eq:C} is replaced with a
%summation as follows:
 %
 \be
 \mat{C}\approx \hat{\mat{C}} = \frac{1}{N}\sum\limits_{i=1}^{N} 
 (\nabla_{\vec{\xi}}f(\vec{\xi}_i))(\nabla_{\vec{\xi}}f(\vec{\xi}_i))^\top
 = \hat{\mat{W}}\hat{\mat{\Lambda}}\hat{\mat{W}}^\top
\label{eq:chat}
 \ee
 %
Clearly, the amount of computational effort associated with constructing the matrix,
$\hat{\mat{C}}$ scales with the number of samples, $N$. Hence, an iterative
computational approach is adopted in this work to gradually increase  
$N$ until the dominant eigenpairs are approximated
with sufficient accuracy; see Section~\ref{sec:method}. 

%%It has been shown that the accuracy of approximated dominant eigenspace,
%%$\hat{\vec{w}}_1$ is inversely proportional to the difference between the
%%smallest eigenvalue in $\hat{\vec{\Lambda}}_1$ and the largest eigenvalue in
%%$\hat{\vec{\Lambda}}_2$~\cite{Constantine:2014}. 
%%Components of the eigenvectors in the active subspace could be used for
%%estimating the so-called activity scores as a measure for global sensitivity
%%and also be used for approximating the DGSMs as discussed in the following
%%section.
  

\section{GSA measures and their links with active subspaces}
\label{sub:gsa}
Consider a function $f = f(\xi_1, \xi_2, \ldots, \xi_\Np)$. 
While the active subspace framework described above does not make any assumptions
about statistical independence of the inputs $\xi_i$, $i = 1, \ldots, \Np$, 
the classical 
framework of variance based sensitivity analysis~\cite{Sobol:2001, Saltelli:2010} 
assumes that the inputs
are statistically independent. While extensions to the cases 
of correlated inputs exist~\cite{Borgonovo:2007,Li:2010,Jacques:2006,Xu:2007,Hart:2017},
 we limit the discussion in this section to the
case of random inputs that are statistically independent and are 
either uniformly distributed or
distributed according to a Boltzmann probability distribution.
Note that a measure $\mu$
on $\R$ is referred to as a Boltzmann measure if it is 
absolutely continuous with respect to the Lebesgue measure  
and admits a density  of the form $\pi(x) = C \exp\{-V(x)\}$,
where $V$ is a continuous function and $C$ a normalization 
constant~\cite{Lamboni:2013}.


The total Sobol' index ($T_i(f)$) of a model output, $f(\vec\xi)$ quantifies
the total contribution of the input, $\xi_i$ to the variance of the
output~\cite{Sobol:2001}. Mathematically, this can be expressed as follows:
%
\be
T_i(f) = 1 - 
\frac{\V[\mathbb{E}(f|\vec{\xi}_{\sim i})]}{\V(f)},
\label{eq:total}
\ee
%
where $\vec{\xi}_{\sim i}$ is the input parameter vector with the  
$i^\text{th}$ entry removed, and $\V$ denotes the variance. The total Sobol' index accounts
for the contribution of a given input to the variability in the output by itself
as well as due to its interaction or coupling with other inputs. 
Determining accurate estimates of $T_i(f)$ typically involves tens of
thousands of model runs and is therefore prohibitive in the case of
compute-intensive applications. Derivative based 
global sensitivity measures (DGSMs)~\cite{Sobol:2009} provide a means for
approximating informative upper bounds on $T_i(f)$ with fewer computations; see 
also~\cite{Vohra:2018}. 

For $f: \Omega \to \R$, we consider the DGSMs,
\[
    \nu_i(f) := \E{\left(\frac{\partial f}{\partial\xi_i}\right)^2} =
                  \int_\Omega 
                  \left(\frac{\partial f}{\partial\xi_i}\right)^2
                  \pi(\vec{\xi})d\vec{\xi}, \quad i = 1, \ldots, \Np.   
\]
Here $\pi$ is the joint PDF of $\vec\xi$. 
Let $\mat{C}$ be as in~\eqref{eq:C}, and consider the spectral decomposition written 
as $\mat{C} = \sum_{k=1}^\Np \lambda_k \vec{w}_k \vec{w}_k^\top$. Herein, we use the notation 
$\ip{\cdot}{\cdot}$ for the Euclidean inner product.
% where
%$\lambda_i$ are the (non-negative) eigenvalues of $\mat{C}$, in descending
%order, and
%$\vec{w}_k$ are the corresponding (orthonormal) eigenvectors. 
The following
result provides a representation of DGSMs in terms of the 
spectral representation of $\mat{C}$: 
\begin{lemma}
We have
$\nu_i(f) = \sum_{k=1}^\Np \lambda_k \ip{\vec{e}_i}{\vec{w}_k}^2$.
\end{lemma}
\begin{proof}
Note that $\nu_i(f) = \vec{e}_i^\top \mat{C} \vec{e}_i$,  
where $\vec{e}_i$ is the $i$th coordinate vector in $\R^\Np$, $i = 1, \ldots, \Np$.
Therefore,
\[
\nu_i(f) = \vec{e}_i^T \Big(\sum_{k=1}^\Np \lambda_k \vec{w}_k \vec{w}_k^\top\Big) \vec{e}_i
 = \sum_{k=1}^\Np \lambda_k \ip{\vec{e}_i}{\vec{w}_k}^2. \qedhere 
\]
\end{proof}
In the case where the eigenvalues decay rapidly to zero, we can obtain
accurate approximations of $\nu_i(f)$ by truncating the summation: 
\newcommand{\act}[3]{\nu_{{#2},{#3}}({#1})}
\newcommand{\actt}[3]{\tilde{\nu}_{{#2},{#3}}({#1})}
\[
   \act{f}{i}{r} =  \sum_{k=1}^r \lambda_k \ip{\vec{e}_i}{\vec{w}_k}^2,
   \quad i = 1, \ldots, \Np, \quad r \leq \Np.
\]
The quantities $\act{f}{i}{r}$ are called activity scores
in~\cite{Diaz:2016,Constantine:2017}.
%The activity scores connect ideas from active subspaces and global sensitivity
%analysis, and can be used to approximate DGSMs.  
The following result, which
can also be found in~\cite{Diaz:2016,Constantine:2017}, quantifies the error in this
approximation. We provide a short proof for completeness. 
\begin{proposition}\label{prp:dgsm_bound} 
For $1 \leq r < \Np$,
\[
0 \leq \nu_i(f) - \act{f}{i}{r} \leq \lambda_{r+1}, \quad i = 1, \ldots, \Np.
\] 
\end{proposition}
\begin{proof} 

%Using the spectral representation of the DGSMs and the definition of activity
%scores we clearly see:
%\[
%\alpha_i(f;r) = \sum_{k=1}^{r}\lambda_k \langle \vec{e}_i, \vec{w}_k \rangle^2 \leq \sum_{k=1}^{N_p}\lambda_k \langle \vec{e}_i, \vec{w}_k \rangle^2 = \nu_i(f), \quad \quad i = 1,\ldots,N_p, \quad r \leq N_p
%\]
%In other words
%\[
%0 \leq  \sum_{k=1}^{N_p}\lambda_k \langle \vec{e}_i, \vec{w}_k \rangle^2 - \sum_{k=1}^{r}\lambda_k \langle \vec{e}_i, \vec{w}_k \rangle^2,  \quad \quad i = 1,\ldots,N_p
%\]
%with equality if $N_p=r$.
%\newline
%We can write:
%\[
%\begin{aligned}
%\nu_i(f) = \sum_{k=1}^{N_p}\lambda_k \langle \vec{e}_i, \vec{w}_k \rangle^2 = \sum_{k=1}^{r}\lambda_k \langle \vec{e}_i, \vec{w}_k \rangle^2 + \sum_{k=r+1}^{N_p}\lambda_k \langle \vec{e}_i, \vec{w}_k \rangle^2  \\
%= \alpha_i(f;r) + \sum_{k=r+1}^{N_p}\lambda_k \langle \vec{e}_i, \vec{w}_k \rangle^2 \leq \alpha_i(f;r) + \lambda_{r+1} \sum_{k=r+1}^{N_p} \langle \vec{e}_i, \vec{w}_k \rangle^2,  \quad \quad i = 1,\ldots,N_p
%\end{aligned}
%\]
%The eigenvectors $\vec{w}_k$ are orthonormal so they all have length 1. Also note that for every $x \in \R^n$ we have \[ \norm{x}^2 = \sum_{k=1}^{n} \langle \vec{x}, \vec{w}_k \rangle^2\] This is known as Parseval's identity.
%In particular in this case $\vec{x} = \vec{e}_i \in \R^{N_p}$ so
%\[1 = \norm{\vec{e}_i}^2 = \sum_{k=r+1}^{N_p} \langle \vec{e}_i, \vec{w}_k \rangle^2\]
%Finally we write:
%\[
%\nu_i(f) \leq \alpha_i(f;r) + \lambda_{r+1}, \quad \quad i = 1,\ldots,N_p
%\]

Note that, $\nu_i(f) - \act{f}{i}{r}= \sum_{k=r+1}^\Np \lambda_k \ip{\vec{e}_i}{\vec{w}_k}^2 \geq 0$,
which gives the first inequality. To see the upper bound, we note,
\[
   \sum_{k=r+1}^\Np \lambda_k \ip{\vec{e}_i}{\vec{w}_k}^2 \leq \lambda_{r+1} \sum_{k=r+1}^\Np \ip{\vec{e}_i}{\vec{w}_k}^2
   \leq \lambda_{r+1}. 
\]
The last inequality holds because 
$1 = \|\vec{e}_i\|_2^2 = 
\sum_{k = 1}^\Np \ip{\vec{e}_i}{\vec{w}_k}^2 
\geq \sum_{k=r+1}^\Np \ip{\vec{e}_i}{\vec{w}_k}^2$.
\end{proof} 
The utility of this result is realized in problems with 
high-dimensional parameter spaces in which 
the eigenvalues $\lambda_i, i=1,\ldots,N_p$, decay rapidly to zero; in 
such cases, this result implies that  $\nu_i(f) \approx \act{f}{i}{r}$,
where $r$ is the \emph{numerical rank} of $\mat{C}$.  This will be especially
effective if there is a large gap in the eigenvalues.  


It was shown in~\cite{Lamboni:2013} that the total Sobol' 
index $T_i(f)$ can be bounded in terms of $\nu_i(f)$:
\begin{equation}\label{equ:sobol_bound}
T_i(f) \leq \frac{C_i}{\V(f)}\nu_i(f), \quad i = 1, \ldots, \Np,
\end{equation}
where for each $i$, $C_i$ is an appropriate \emph{Poincar\'{e}} constant
that depends on the distribution of $\xi_i$; see~\cite{Lamboni:2013}.
For instance, if $\xi_i$ is uniformly distributed on $[-1, 1]$, then $C_i = 2/\pi^2$; and in the 
case $\xi_i$ is normally distributed with variance $\sigma_i^2$, then $C_i = \sigma_i^2$. 
The bound~\eqref{equ:sobol_bound} provides a strong theoretical basis for using DGSMs to identify 
unimportant inputs. 

Combining Proposition~\ref{prp:dgsm_bound} and~\eqref{equ:sobol_bound}, shows
an interesting link between the activity scores and total Sobol' indices.
Specifically, by computing the activity scores, we can identify the unimportant
inputs.  
One can attempt to reduce parameter dimension by fixing
unimportant inputs at nominal values. Suppose $\xi_i$ is
deemed unimportant as a result of a global sensitivity analysis (GSA)
based on the activity scores; we want to estimate
the approximation error that occurs once $\xi_i$ 
is fixed at a nominal value.
%Let $\I \subset \{1, \ldots,
%Np\}$ be an index set that indexes the unimportant parameters in the parameter
%vector $\vec{\xi}$.  Letting $|\I|$ denote the number of elements of $\I$, 
%we denote by $\vec{z} \in \R^{|\I|}$ a generic vector of nominal values for
%the unimportant parameters $\vec\xi_\I$, and let $\vec{y}$ be a generic vector of
%important variables, $\vec\xi_{\I^c}$,
%where $\I^c$ denotes the complement of $\I$ in $\{1, \ldots, \Np\}$. 
%For a vector $\xi$, we let $\xin$ denote the vector obtained by removing
%$\xi_i$. 

Let $\vec\xi$ be given and let $z$ be a nominal value for $\xi_i$.  
%we define 
%$\vec{y}^z(\vec\xi)$ as the vector with entries $y^z_j = \xi_j$ for $j \neq i$
%and $y^z_i = z$.
We consider the \emph{reduced} model, 
obtained by fixing $\xi_i$ at a nominal value $z \in \R$: 
\[
f^{z}(\vec\xi) = f(\xi_1, \xi_2, \ldots, \xi_{i-1}, z, \xi_{i+1}, \ldots, \xi_\Np),
\] 
and consider the following error indicator:
\[
\mathcal{E}(z) =
\frac{ \int_\Omega \big( f(\vec\xi) - f^{z}(\vec\xi)\big)^2 \, \mu(d\vec\xi) }
          {\int_\Omega f(\vec\xi)^2 \, \mu(d\vec\xi)}.
\] 
\begin{theorem}
We have $\Ez{ \mathcal{E}(z)} \leq 2C_i\big(\act{f}{i}{r} + \lambda_{r+1}\big)/{\V(f)}$, 
for $1 \leq r < \Np$.
\end{theorem}
%
%\[
%       T_i(f) \leq \frac{C_i}{\V(f)}(\act{f}{i}{r} + \lambda_{r+1}).
%\]
%we obtain the following useful inequality, which
%links activity scores and Sobol' indices:
%for $1 \leq r \leq \Np$, and $i = 1, \ldots, \Np$, we have
%\[
%       T_i(f) \leq \frac{C_i}{\V(f)}(\act{f}{i}{r} + \lambda_{r+1}).
%\]
%This was noted in~\cite{Diaz:2016} for the specific case of all $\xi_i$'s being
%iid $U(0, 1)$, in which case $C_i = 1/\pi^2$, $i = 1, \ldots, \Np$.
\begin{proof} 
Note that, since 
$\int_\Omega f(\vec\xi)^2 \, \mu(d\vec\xi) = \V(f) + 
\left(\int_\Omega f(\vec\xi) \, \mu(d\vec\xi)\right)^2 \geq \V(f)$, we have
\[
\Ez{ \mathcal{E}(z)} \leq \frac{1}{\V(f)} \Ez{ 
\int_\Omega \big( f(\vec\xi) - f^{z}(\vec\xi)\big)^2 \, \mu(d\vec\xi)}
= 2 T_i(f), 
\]
where the equality can be shown using arguments similar to the proof of the main result 
in~\cite{SobolTarantolaGatelliEtAl07}. Using this, along with~\eqref{equ:sobol_bound} and
Proposition~\ref{prp:dgsm_bound}, we have 
\[
\Ez{ \mathcal{E}(z)} \leq 
\frac{2C_i}{\V(f)}\nu_i(f)
\leq 
\frac{2C_i}{\V(f)}\big[\act{f}{i}{r} + \lambda_{r+1}\big]. \qedhere
\]
\end{proof}

In~\cite{Vohra:2018} the following normalized screening metric
\[
   \tilde{\nu}_i(f) = \frac{C_i \nu_i(f)}{\sum_{i=1}^\Np C_i \nu_i(f)},
\]
was shown to be useful for detecting unimportant inputs. 
We can also bound the normalized DGSMs using activity scores as follows. It is straightforward to see  
that
\[
\tilde{\nu}_i(f) \leq 
\frac{ C_i \big(\act{f}{i}{r} + \lambda_{r+1}\big)}{\sum_{i=1}^\Np C_i \act{f}{i}{r}}
=\frac{C_i \act{f}{i}{r}}{\sum_{i=1}^\Np C_i \act{f}{i}{r}} + \kappa_i \lambda_{r+1}, 
\]
with $\kappa_i = C_i / (\sum_i C_i \act{f}{i}{r})$. 
In the case where where $\lambda_{r+1} \approx 0$, 
this motivates definition of
normalized activity scores
\[
   \actt{f}{i}{r} =  \frac{C_i \act{f}{i}{r}}{\sum_{i=1}^\Np C_i \act{f}{i}{r}}.
\] 

\begin{remark}
Note that if $\xi_i$ are iid, the $C_i$'s in the definition of the 
normalized screening metric will cancel and 
\[
    \tilde{\nu}_i(f) = \frac{\nu_i(f)}{\sum_{i=1}^\Np \nu_i(f)} 
      = \frac{\vec{e}_i^T \mat{C} \vec{e}_i}{\trace(\mat{C})} 
      = \frac{\sum_{k=1}^\Np \lambda_k \ip{\vec{e}_i}{\vec{w}_k}^2}{\sum_{k = 1}^\Np \lambda_k}.
\]
In this case, the normalized activity scores reduce to
\[
   \actt{f}{i}{r} =  \frac{\act{f}{i}{r}}{\sum_{i=1}^\Np \act{f}{i}{r}}.
\]
\end{remark}
