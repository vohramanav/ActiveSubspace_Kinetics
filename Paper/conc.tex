\section{Summary and Discussion}
\label{sec:conc}

%%Brief summary
%-Mathematical connections between Sobol, DGSMs, and activity scores exist and could
%be exploited to perform GSA with reduced effort.
%-Iterative implementation yields advantage
%-Consistency between the two approaches with DGSM as well as among themselves
%encourages the use of gradient-free.
%-Grad-free was found to be suitable for a high-dimensional problem.
%-Based on the results presented, the following parameter were found to be 
%significant.
%
%discussion
%- Methodology is agnostic to the choice of mechanism and could be applied to other
%systems.
%- However, constraints exist in terms of differentiability of the QoI. Not all
%input/output relationships exhibit an active subspace. However, it is important to
%check.
%- Activity scores could be used to identify key reactions and thereby construct
%reduced order mechanisms for reducing computational effort and investigating the
%inadequacy associated with rate laws.
 
In this work, we focused on the uncertainty associated with the
rate-controlling parameters in the H$_2$/O$_2$ reaction mechanism and its
impact on ignition delay predictions. The mechanism involves 19 different
reactions and in each case, the reaction rate depends upon the choice of a
pre-exponent and an activation energy. Hence, in theory, the evolution of the
chemical system depends upon 38 inputs. In this work, we have considered 
epistemic uncertainty in all pre-exponents and activation energies with non-zero
nominal values. Hence, the input space dimensionality is 33 instead of 38.  
%Conventional means for uncertainty quantification such as those involving
%surrogate model construction as well as sensitivity analysis would thus be
%computationally challenging.
To facilitate efficient uncertainty analysis, we focused our efforts on
reducing the dimensionality of the problem by identifying important directions
in the parameter space such that the variability in the model output is
predominantly captured along these directions. This was done through active
subspace construction.  Additionally, we demonstrated that the activity scores,
computed using the components of the dominant eigenvectors provide an efficient
means for approximating derivative based global sensitivity measures (DGSMs).
Furthermore, we established generalized mathematical linkages between the
different global sensitivity measures: activity scores, DGSMs, and total Sobol'
index which could be exploited to reduce computational effort associated with
global sensitivity analysis. 
 

Two computational strategies were explored to compute the active subspace for
the H$_2$/O$_2$ kinetics application in this work. The first strategy, referred
to as perturbation-based, used finite differences to estimate the gradient of the
model output with respect to the uncertain inputs. The second strategy, referred
to as regression-based, involved gradient estimation by means of approximating
the available set of model evaluations using a linear regression fit.  The two
strategies were shown to yield consistent results for the 19-dimensional
problem whereby only the pre-exponents were considered to be uncertain.
Additionally, the activity scores were also shown to be consistent with the
screening metric estimates based on DGSMs in~\cite{Vohra:2018}. An
iterative procedure was adopted to enhance the computational efficiency
associated with both strategies. 

The two strategies were further implemented to discover and compute the active
subspace for the 33-dimensional problem. 
Both strategies revealed the existence
of a 1-dimensional active subspace. The activity scores based on the dominant
eigenvector and the total Sobol' indices based on the 1-dimensional surrogate in
the subspace, computed using the perturbation-based strategy were found to be
consistent with each other. Both sensitivity measures indicated that the
uncertainty in the ignition delay is largely dependent on $A_1$, $A_9$,
$E_{a,1}$ and $E_{a,15}$, although contributions from $A_{15}$ and $A_{17}$
were also found to be significant. 

Computational gains using the proposed strategy in this work are essentially
realized by using the 1-dimensional surrogate in the active subspace that
essentially casts a mutlivariate input in the full space into a univariate
input in the reduced space. The surrogate can be used to accelerate forward
propagation of the uncertainty as well as parameter estimation in a Bayesian
setting. Hence, we focused our efforts to verify the accuracy of the
1-dimensional surrogates, obtained using both strategies.  The 1-dimensional
surrogates were assessed for accuracy by first evaluating the relative $L^2$-norm
of the error between its predictions and model evaluations at $10^4$
independent MC samples in the 33-dimensional input domain. The relative error
based on predictions from the surrogate, computed using the regression-based
strategy were found to be an order of magnitude larger than that estimated
using the surrogate from the perturbation-based strategy.

Additionally, the surrogates were assessed by comparing probability
distributions for the ignition delay,
constructed using surrogate predictions and model evaluations at the same set
of 10$^4$ random samples. The PDFs based on model evaluations and the
1-dimensional surrogate from the perturbation-based strategy were found to be
almost identical. However, the PDF based on 1-dimensional surrogate from the
regression-based strategy although captured the modal estimate with reasonable
accuracy, was observed to underestimate the uncertainty in the ignition delay.
Mean estimates of the ignition delay using model evaluations and predictions
using the two surrogates at 10$^4$ samples were found to be in agreement.
However, the standard deviation, computed using predictions from
the surrogate based on the regression-based strategy was found to be accurate
only upto the first significant digit. Moreover, it was observed that the
activity scores, computed using the regression-based strategy failed to capture
the sensitivity towards one of the most important parameters, $E_{a,15}$.  We attribute
this discrepancy to two potential sources of numerical errors: (1)
Regression-based approximation of the gradient of the model output, and (2)
Scatter in the corresponding SSP (see Figure~\ref{fig:hd}) leading to numerical
error incurred by the linear regression fit in this case. Hence, while the
regression-based approach 
reduces computational effort, the present investigations clearly reveal
potential pitfalls associated with this approach. 
%
%More specifically, using
%the regression-based approach, 
%\begin{itemize} 
%\item The desired statistics (e.g. mean and variance) of the
%QoI might not be approximated to a desired level of accuracy.  
%\item The
%approach might fail to capture the sensitivity  of the QoI towards all inputs.
%\end{itemize} 
Therefore, in general perturbation-based approaches are preferred. The
regression-based approach can be explored in situations involving intensive
simulations where the perturbation-based methods are intractable, and the goal
is obtaining rough estimates of the statistics of the QoI as opposed to a more
detailed analysis such as those involving parametric sensitivity.  We also
mention that alternate regression based approaches such as ones based on
computing a global quadratic model have been proposed and used in the
literature; see e.g.,~\cite{Constantine:2017a}. The applicability of such an
approach in the context of high-dimensional chemical reaction networks is
subject to future work. 

The computational framework presented in this work is agnostic to the choice of
the chemical system and can be easily adapted for other systems as long
as the quantity of interest is continuously differentiable in the considered
domain of the inputs.  We have demonstrated that the active subspace could be
exploited for efficient forward propagation of the uncertainty from inputs to
the output. However, the activity scores and the low-dimensional surrogate
could further guide optimal allocation of computational resources for
calibration of the important rate-controlling parameters in a Bayesian setting.
Additionally, dimension reduction using active subspaces could assist in
developing robust formulations for predicting discrepancy between simulations
and measurements due to epistemic uncertainty in the model inputs.
