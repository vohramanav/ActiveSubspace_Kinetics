\section{Summary and Discussion}
\label{sec:conc}

%%Brief summary
%-Mathematical connections between Sobol, DGSMs, and activity scores exist and could
%be exploited to perform GSA with reduced effort.
%-Iterative implementation yields advantage
%-Consistency between the two approaches with DGSM as well as among themselves
%encourages the use of gradient-free.
%-Grad-free was found to be suitable for a high-dimensional problem.
%-Based on the results presented, the following parameter were found to be 
%significant.
%
%discussion
%- Methodology is agnostic to the choice of mechanism and could be applied to other
%systems.
%- However, constraints exist in terms of differentiability of the QoI. Not all
%input/output relationships exhibit an active subspace. However, it is important to
%check.
%- Activity scores could be used to identify key reactions and thereby construct
%reduced order mechanisms for reducing computational effort and investigating the
%inadequacy associated with rate laws.
 
 In this work, we focused on the uncertainty associated with the rate-controlling parameters in
 the H$_2$/O$_2$ reaction mechanism and its impact on ignition delay predictions. The mechanism
 involves 19 different reactions and in each case, the reaction rate depends upon the choice of
 a pre-exponent and an activation energy. Hence, in theory, the evolution of the chemical system
 depends upon 38 inputs which is reasonably high-dimensional. Conventional means
 for uncertainty quantification such as those involving a surrogate model construction as well as
 sensitivity analysis would thus be computationally challenging and even prohibitive in some cases. 
 Hence, we focused our efforts on reducing the dimensionality of the problem by identifying important
 directions (active subspaces) in the parameter space such that the variability in the model output is 
 predominantly captured along these directions. Additionally, we demonstrated that the activity scores,
 computed using the components of the dominant eigenvectors can be used to approximate the
 DGSMs. Furthermore, we established generalized mathematical linkages between the different global sensitivity measures:
 activity scores, DGSMs, and total Sobol' index which could be exploited to reduce computational effort 
 associated with global sensitivity analysis. 
 
Two computational strategies were explored to compute the active subspace for the H$_2$/O$_2$
kinetics application in this work. The first strategy referred to as gradient-based requires computation of gradient of the 
model output with respect to the input, whereas, the second strategy, referred to as gradient-free implements a local
linear approximation to the model output in order to estimate the gradient using a regression-based fit.
The two strategies were shown to yield
consistent results for the 19-dimensional problem whereby only the pre-exponents were considered to be uncertain.
Additionally, the activity scores were also shown to be consistent with the screening metric estimates based on
DGSMs in~\cite{Vohra:2018}. Note that an iterative procedure was adopted to enhance the computational efficiency
associated with both strategies. 

Both strategies were further implemented to discover and compute the active
subspace for the 33-dimensional problem. The gradient-based approach yielded a 1-dimensional active subspace
and the activity scores which revealed that the uncertainty in the ignition delay is largely dependent on $A_1$, $A_9$, 
$E_{a,1}$ and $E_{a,15}$, although contributions from $A_{15}$ and $A_{17}$ were also found to be significant. The activity
scores were verified against the total Sobol' indices, computed using the 1-dimensional surrogate fit to the model response in
the subspace. The 1-dimensional surrogate was verified for accuracy by evaluating the relative L-2
norm of the error between its predictions and model evaluations at $10^4$ samples in the 33-dimensional input domain.
Additionally, the surrogate was verified by comparing probability distributions (based on kernel density estimation) for
the ignition delay based on its predictions and model evaluations. The two PDFs were observed to be nearly identical
and consequently, their mean and standard deviations were found to be in excellent agreement. On the other hand,
the gradient-free approach failed to capture the sensitivity towards the most important parameter, $E_{a,15}$. 
We attribute this discrepancy to two potential sources of numerical errors: (1) Regression-based approximation of the
gradient of the model output, and (2) Scatter in the corresponding SSP (see Figure~\ref{fig:hd}) leading to numerical
error incurred by the linear regression fit in this case. The PDF was observed to be consistent with the true PDF in its
estimate for the mode and the mean value of the ignition delay. However, it was observed to underestimate the uncertainty 
associated with the ignition delay. Hence, the gradient-free approach might prove to be valuable in applications 
involving intensive simulations (rendering the gradient-based approach prohibitive), and focused on a reasonable 
approximation of the mean value and the uncertainty associated with the quantity of interest. Based on our findings, the 
estimate for the mean is expected to be relatively more accurate. 
 
 The authors would like to highlight that the computational strategy presented in this work is agnostic to the choice
 of the chemical system and could thus be easily adapted for other systems. However, it is important to note that
 its applicability depends upon the nature of the relationship between the model output and the inputs. Specifically,
 the model output is required to be differentiable throughout the considered input domain to be able to evaluate
 the active subspace with reasonable accuracy. Hence, it is a realistic possibility that an active subspace might not
 exist for a given input-output relationship. While the active subspace could be exploited for a forward propagation of
 the uncertainty from inputs to the output, the low-dimensional surrogate could further be used to accelerate 
 calibration of the rate-controlling parameters in a Bayesian setting. Moreover, the activity scores could help
 identify important stages in a complex reaction mechanism and thereby guide development of a simplified
 mechanism. Furthermore, dimension reduction using active subspaces could be useful in studies aimed at
 investigating the inadequacy or model error associated with the reaction rate laws for such systems.  
 
 
 
 
 
 
 
