\section{Summary and Discussion}
\label{sec:conc}

%%Brief summary
%-Mathematical connections between Sobol, DGSMs, and activity scores exist and could
%be exploited to perform GSA with reduced effort.
%-Iterative implementation yields advantage
%-Consistency between the two approaches with DGSM as well as among themselves
%encourages the use of grad-free.
%-Grad-free was found to be suitable for a high-dimensional problem.
%-Based on the results presented, the following parameter were found to be 
%significant.
%
%discussion
%- Methodology is agnostic to the choice of mechanism and could be applied to other
%systems.
%- However, constraints exist in terms of differentiability of the QoI. Not all
%input/output relationships exhibit an active subspace. However, it is important to
%check.
%- Activity scores could be used to identify key reactions and thereby construct
%reduced order mechanisms for reducing computational effort and investigating the
%inadequacy associated with rate laws.
 
 In this work, we focused on the uncertainty associated with the rate-controlling parameters in
 the H$_2$/O$_2$ reaction mechanism and its impact on ignition delay predictions. The mechanism
 involves 19 different reactions and in each case, the reaction rate depends upon the choice of
 a pre-exponent and an activation energy. Hence, in theory, the evolution of the chemical system
 depends upon 38 inputs which is reasonably high-dimensional. In such case, conventional means
 for uncertainty quantification such as those involving a surrogate model construction as well as
 sensitivity analysis would be computationally challenging and even prohibitive in some cases. 
 Hence, we focused our efforts on reducing the dimensionality of the problem by identifying important
 directions (active subspaces) in the parameter space such that the variability in the model output is 
 predominantly captured along these directions. Additionally, we demonstrate that the activity scores,
 computed using the components of the dominant eigenvectors can be used to approximate the
 DGSMs. Furthermore, we establish mathematical linkages between the different global sensitivity measures:
 activity scores, DGSMs, and total Sobol' index which could be exploited to reduce computational effort 
 associated with global sensitivity analysis. 
 
Two computational strategies were explored to compute the active subspace for the H$_2$/O$_2$
kinetics application in this work. The first strategy referred to as grad-based requires computation of gradient of the 
model output with respect to the input, whereas, the second strategy, referred to as grad-free implements a local
linear approximation to the model output in order to estimate the gradient. The two strategies are shown to yield
consistent results for the 19-dimensional problem whereby only the pre-exponents are considered to be uncertain.
Additionally, the activity scores are also shown to be consistent with the screening metric estimates based on
DGSMs in~\cite{Vohra:2018}. Note that an iterative procedure is adopted for implementing the two strategies
for enhancing computational efficiency. 
Motivated by these findings, the grad-free approach is used for computing the active subspace in the high-dimensional
case involving 33 uncertain parameters. Based on the prior marginals of the uncertain parameters, our results
indicate that the uncertainty in the ignition delay is largely dependent on $A_1$, $A_9$, and $E_{a,1}$, although
 contributions from $A_{15}$ and $A_{17}$ are also found to be significant. 
 
 The authors would like to highlight that the computational strategy presented in this work is agnostic to the choice
 of the chemical system and could thus be easily adapted for other systems. However, it is important to note that
 its applicability depends upon the nature of the relationship between the model output and the inputs. Specifically,
 the model output is required to be differentiable throughout the considered input domain to be able to evaluate
 the active subspace with reasonable accuracy. Hence, it is a realistic possibility that an active subspace might not
 exist for a given input-output relationship. While the active subspace could be exploited for a forward propagation of
 the uncertainty from inputs to the output, the low-dimensional surrogate could further be used to accelerate 
 calibration of the rate-controlling parameters in a Bayesian setting. Moreover, the activity scores could help
 identify important stages in a complex reaction mechanism and thereby guide development of a reduced-order
 surrogate. Furthermore, dimension reduction using active subspaces could be useful in studies aimed at
 investigating the inadequacy or model error associated with the reaction rate laws for such systems.  
 
 
 
 
 
 
 