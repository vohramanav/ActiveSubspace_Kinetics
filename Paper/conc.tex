\section{Summary and Discussion}
\label{sec:conc}

%%Brief summary
%-Mathematical connections between Sobol, DGSMs, and activity scores exist and could
%be exploited to perform GSA with reduced effort.
%-Iterative implementation yields advantage
%-Consistency between the two approaches with DGSM as well as among themselves
%encourages the use of gradient-free.
%-Grad-free was found to be suitable for a high-dimensional problem.
%-Based on the results presented, the following parameter were found to be 
%significant.
%
%discussion
%- Methodology is agnostic to the choice of mechanism and could be applied to other
%systems.
%- However, constraints exist in terms of differentiability of the QoI. Not all
%input/output relationships exhibit an active subspace. However, it is important to
%check.
%- Activity scores could be used to identify key reactions and thereby construct
%reduced order mechanisms for reducing computational effort and investigating the
%inadequacy associated with rate laws.
 
 In this work, we focused on the uncertainty associated with the rate-controlling parameters in
 the H$_2$/O$_2$ reaction mechanism and its impact on ignition delay predictions. The mechanism
 involves 19 different reactions and in each case, the reaction rate depends upon the choice of
 a pre-exponent and an activation energy. Hence, in theory, the evolution of the chemical system
 depends upon 38 inputs which is reasonably high-dimensional. Conventional means
 for uncertainty quantification such as those involving a surrogate model construction as well as
 sensitivity analysis would thus be computationally challenging and even prohibitive in some cases. 
 Hence, we focused our efforts on reducing the dimensionality of the problem by identifying important
 directions (active subspace) in the parameter space such that the variability in the model output is 
 predominantly captured along these directions. Additionally, we demonstrated that the activity scores,
 computed using the components of the dominant eigenvectors can be used to approximate the
 DGSMs. Furthermore, we established generalized mathematical linkages between the different global sensitivity measures:
 activity scores, DGSMs, and total Sobol' index which could be exploited to reduce computational effort 
 associated with global sensitivity analysis. 
 
Two computational strategies were explored to compute the active subspace for the H$_2$/O$_2$
kinetics application in this work. The first strategy referred to as perturbation-based
used finite difference to estimate the gradient of the model output with respect to the uncertain
inputs. The second strategy referred to as regression-based involved gradient estimation by
means of approximating the available set of model evaluations using a linear regression fit. 
The two strategies were shown to yield
consistent results for the 19-dimensional problem whereby only the pre-exponents were considered to be uncertain.
Additionally, the activity scores were also shown to be consistent with the screening metric estimates based on
DGSMs in~\cite{Vohra:2018}. Note that an iterative procedure was adopted to enhance the computational efficiency
associated with both strategies. 

The two strategies were further implemented to discover and compute the active
subspace for the 33-dimensional problem. Both strategies revealed the existence of a 1-dimensional
active subspace. The activity scores based on the dominant eigenvector and the total Sobol indices
based on the 1-dimensional surrogate in the subspace, computed using the 
perturbation-based strategy were found to be consistent with each other. Both sensitivity measures
 indicated that the uncertainty in the ignition delay is largely dependent on $A_1$, $A_9$, 
$E_{a,1}$ and $E_{a,15}$, although contributions from $A_{15}$ and $A_{17}$ were also found to be significant. 

Computational gains using the proposed strategy in this work are essentially realized by using the
1-dimensional surrogate in the active subspace that essentially casts a mutlivariate input in the full space
into a univariate input in the reduced space. The surrogate can be used to accelerate forward propagation
of the uncertainty as well as parameter estimation in a Bayesian setting. Hence, we focused our efforts to
verify the accuracy of the 1-dimensional surrogates, obtained using both strategies.  
The 1-dimensional surrogates were assessed for accuracy by first evaluating the relative L-2
norm of the error between its predictions and model evaluations at $10^4$ independent MC samples in the
33-dimensional input domain. The relative error based on predictions from the surrogate, computed
using the regression-based strategy were found to be an order of magnitude larger than that estimated
using the surrogate from the perturbation-based strategy.

Additionally, the surrogates were assessed by comparing probability distributions (based on kernel density estimation)
for the ignition delay, constructed using surrogate predictions and model evaluations at the same set of 10$^4$
random samples. The PDFs based on model evaluations and the 1-dimensional surrogate from the perturbation-based
strategy were found to be almost identical. However, the PDF based on 1-dimensional surrogate from the
regression-based strategy although captured the modal estimate with reasonable accuracy, was observed to
under-estimate the uncertainty in the ignition delay. Mean estimates of the ignition delay using model evaluations
and predictions using the two surrogates at 10$^4$ samples were found to be in agreement. However, as expected, the
standard deviation, computed using predictions from the surrogate based on the regression-based strategy was
found to be accurate only upto the first significant digit. Moreover, it was observed that the activity scores,
computed using the regression-based strategy failed to capture the sensitivity towards the 
most important parameter, $E_{a,15}$. 
We attribute this discrepancy to two potential sources of numerical errors: (1) Regression-based approximation of the
gradient of the model output, and (2) Scatter in the corresponding SSP (see Figure~\ref{fig:hd}) leading to numerical
error incurred by the linear regression fit in this case. Based on our findings in this work, it can be said that
the regression-based strategy would be useful in situations involving complex mechanisms with a large number of
uncertain inputs thereby rendering the UQ analysis intensive, and in applications where the focus is on a
reasonable approximation of the statistics of the QoI. 
 
 The authors would like to highlight that the computational framework presented in this work is agnostic to the choice
 of the chemical system and could thus be easily adapted for other systems. However, it is important to note that
 its applicability depends upon the nature of the relationship between the model output and the inputs. Specifically,
 the model output is required to be differentiable throughout the considered input domain to be able to evaluate
 the active subspace with reasonable accuracy. Hence, it is a realistic possibility that an active subspace might not
 exist for a given input-output relationship. While the active subspace could be exploited for a forward propagation of
 the uncertainty from inputs to the output, the low-dimensional surrogate could further be used to accelerate 
 calibration of the rate-controlling parameters in a Bayesian setting. Moreover, the activity scores could help
 identify important stages in a complex reaction mechanism and thereby guide development of a simplified
 mechanism. Furthermore, dimension reduction using active subspaces could be useful in studies aimed at
 investigating the inadequacy or model error associated with the reaction rate laws for such systems.  
 
 
 
 
 
 
 
