\section{Introduction}
\label{sec:intro}

%Constributions:

Time evolution of a chemically reacting system is largely dependent upon rate
constants associated with individual reactions. The rate constants are
typically assumed to exhibit a certain correlation with temperature (e.g.,
Arrhenius-type). Hence, accurate specification of the rate-controlling
parameters is critical to the fidelity of simulations. However, in practical
applications, these parameters are either specified using expert knowledge or
estimated based on a regression-fit to a set of sparse and noisy
data~\cite{Burnham:1987, Burnham:1988, Vohra:2011, Sarathy:2012}.
%Consequently, the underlying uncertainty associated with the parameters and
%therefore the simulation is typically ignored. 
Consequently, the inclusion of
uncertainty due to corrupted calibration data, parametric uncertainty,
and the discrepancy between predictions and experiments
due to model inadequacy has been subject of increased research efforts in recent
years~\cite{Vohra:2014, Vohra:2017, Morrison:2018, Hantouche:2018}. 

In complex mechanisms involving a large number of reactions, characterizing the
propagation of uncertainty from a large set of inputs to the model output is
challenging due to the associated computational effort.  A major focus of this
article is the implementation of a robust framework that aims to identify
\emph{important} directions in the input space that predominantly capture the
variability in the model output. These directions, which span the so called
\emph{active subspace}~\cite{Constantine:2015}, are the dominant eigenvectors
of a matrix derived from the gradient information of the model output with
respect to an input. The active subspace methodology thus focuses on reducing
the dimensionality of the problem, and hence the computational effort
associated with uncertainty propagation. 

The application problem considered in this work is the
H$_2$/O$_2$ reaction mechanism from~\cite{Yetter:1991}. This mechanism has gained
a lot of attention as a potential source of clean energy for
locomotive applications~\cite{Das:1996}, and more recently in fuel
 cells~\cite{Loges:2008,Cosnier:2016}. 
The mechanism involves 19
reactions including chain reactions, dissociation/recombination reactions, and
formation and consumption of intermediate species; see Table~\ref{tab:kinetics}. 
For each reaction, the reaction rate is assumed to follow an Arrhenius
correlation with temperature:
%
\be
k_i(T) = A_iT^{n_i}\exp(-E_{a,i}/RT), 
\label{eq:rate}
\ee
%
where $A_i$ is the pre-exponent, $n_i$ is the index of $T$, $E_{a,i}$ is the
activation energy corresponding to the $i^{th}$ reaction, and $R$ is the
universal gas constant.  
%
\begin{table}[htbp]
\renewcommand*{\arraystretch}{0.9}
\begin{center}
\begin{tabular}{llll}
\toprule
Reaction \#     & Reaction &&\\
\bottomrule
$\mathcal{R}_1$ & H + O$_2$          & $\rightleftharpoons$ & O + OH \\
$\mathcal{R}_2$ & O + H$_2$          & $\rightleftharpoons$ & H + OH \\
$\mathcal{R}_3$ & H$_2$ + OH         & $\rightleftharpoons$ & H$_2$O + H \\
$\mathcal{R}_4$ & OH + OH            & $\rightleftharpoons$ & O + H$_2$O \\
$\mathcal{R}_5$ & H$_2$ + M          & $\rightleftharpoons$ & H + H + M \\
$\mathcal{R}_6$ & O + O + M          & $\rightleftharpoons$ & O$_2$ + M \\
$\mathcal{R}_7$ & O + H + M          & $\rightleftharpoons$ & OH + M \\
$\mathcal{R}_8$ & H + OH +M          & $\rightleftharpoons$ & H$_2$O + M \\
$\mathcal{R}_9$ & H + O$_2$ + M      & $\rightleftharpoons$ & HO$_2$ + M \\
$\mathcal{R}_{10}$ & HO$_2$ + H      & $\rightleftharpoons$ & H$_2$ + O$_2$ \\
$\mathcal{R}_{11}$ & HO$_2$ + H      & $\rightleftharpoons$ & OH + OH \\
$\mathcal{R}_{12}$ & HO$_2$ + O      & $\rightleftharpoons$ & O$_2$ + OH \\
$\mathcal{R}_{13}$ & HO$_2$ + OH     & $\rightleftharpoons$ & H$_2$O + O$_2$ \\
$\mathcal{R}_{14}$ & HO$_2$ + HO$_2$ & $\rightleftharpoons$ & H$_2$O$_2$ + O$_2$ \\
$\mathcal{R}_{15}$ & H$_2$O$_2$ + M  & $\rightleftharpoons$ & OH + OH + M \\
$\mathcal{R}_{16}$ & H$_2$O$_2$ + H  & $\rightleftharpoons$ & H$_2$O + OH \\
$\mathcal{R}_{17}$ & H$_2$O$_2$ + H  & $\rightleftharpoons$ & HO$_2$ + H$_2$ \\
$\mathcal{R}_{18}$ & H$_2$O$_2$ + O  & $\rightleftharpoons$ & OH + HO$_2$ \\
$\mathcal{R}_{19}$ & H$_2$O$_2$ + OH & $\rightleftharpoons$ & HO$_2$ + H$_2$O \\
\bottomrule
\end{tabular}
\end{center}
\caption{Reaction mechanism for H$_2$/O$_2$ from~\cite{Yetter:1991}}.
\label{tab:kinetics}
\end{table}
%
The global reaction associated with the H$_2$/O$_2$ mechanism can
be considered as follows:
\be
2\text{H}_2 + \text{O}_2 \rightarrow 2\text{H}_2\text{O}.
\label{eq:global}
\ee 
The equivalence ratio ($\phi$) is given as follows:
%
\be
\phi = \frac{(M_{\text{H}_2}/M_{\text{O}_2})_\text{obs}}{(M_{\text{H}_2}/M_{\text{O}_2})_\text{st}}.
\label{eq:phi}
\ee
%
where the numerator on the right-hand-side denotes the ratio of the fuel (H$_2$)
and oxidizer (O$_2$) at a given condition to the same quantity under stoichiometric
conditions. In this study, computations were performed at fuel-rich conditions,
$\phi$ = 2.0. Homogeneous ignition at constant pressure is simulated using the
TChem software package~\cite{Safta:2011}, and the time required for the rate of
temperature increase to exceed a given threshold, regarded as \emph{ignition delay}
is recorded. 

We seek to understand the impact of uncertainty in the
rate-controlling parameters, pre-exponent~($A_i$) and the activation
energy~($E_{a,i}$) on the ignition delay. The $A_i$'s associated with all
reactions and the $E_{a,i}$'s with non-zero nominal estimates
are considered to be uniformly distributed about their nominal estimates provided
in~\cite{Yetter:1991}.  The total number of uncertain inputs is 33 which makes
the present problem computationally challenging due to the large number of 
uncertain parameters.  
To address this challenge, we focus on reducing the dimensionality
of the problem by computing the aforementioned active subspace.
This involves repeated evaluations of the gradient of a model output with
respect to the input parameters. Estimating the
gradients can however be computationally challenging in high-dimensional settings. 
Hence, in this work, we further explore the suitability of a gradient-free
approach that involves a regression-based local linear approximation of
the model output as discussed later in~\ref{sub:gradfree}. The two approaches
 are based on
algorithms 1.1 and 1.2 in~\cite{Constantine:2015}. Both strategies
are implemented in an iterative manner i.e. a new set of model evaluations are
generated, and gradient evaluations are performed at each iteration until 
a reasonably accurate approximation to the active subspace is achieved, as
discussed later in Section~\ref{sec:method}. 

An alternate approach to dimension reduction involves computing the global
sensitivity measures associated with uncertain inputs of a model. Depending
upon the estimates of the sensitivity measures, only the important inputs are
varied for the purpose of uncertainty quantification (UQ). Sobol' indices are
commonly used as global sensitivity measures~\cite{Sobol:2001}. However,
obtaining converged estimates of the Sobol' indices typically requires tens of
thousands of model runs and is therefore impractical in situations involving
expensive model simulations and/or high-dimensional input parameter spaces.
Recently, the derivative-based global sensitivity measures~(DGSMs) were
developed to approximate upper bound on the Sobol' indices with much fewer
computations~\cite{Sobol:2009, Lamboni:2013}. It was noted
in~\cite{Diaz:2016,Constantine:2017} that DGSMs can be approximated by
exploiting their links with active subspaces. This led to definition of the so
called \emph{activity scores}. In Section~\ref{sub:gsa}, we build on these
ideas to provide a complete analysis of links between Sobol indices, DGSMs, and
activity scores for functions of independent random inputs whose distribution
law belongs to the class of Boltzmann probability measures. 
It is worth mentioning that computing global sensitivity measures provides 
important information about a model that go beyond dimension reduction. By 
identifying parameters with significant impact on model uncertainty global 
sensitivity analysis can guide model calibration and risk assessment. 
 
The main contributions of this paper are as follows: 
\begin{itemize}
\item 
Active subspace
discovery in a high-dimensional H$_2$/O$_2$ kinetics problem involving 33
uncertain inputs. The implementation of the methodology presented in this work
resulted in a 1-dimensional active subspace, indicating immense potential
for computational savings. The presented
analysis can also guide practitioners in area of chemical kinetics on using the
method of active subspaces as an efficient means for uncertainty propagation.  
\item Comprehensive numerical investigation of the gradient-based and gradient-free approach: 
we investigate the suitability of both approach
for computing active subspaces, surrogate modeling, and approximating global
sensitivity measures through a comprehensive set of numerical experiments. Our
results reveal insight into the merits of the methods approach as well as
their shortcomings.  
\item Analysis of the links between  
global sensitivity measures: 
By connecting the recent advances in theory of variance based and
derivative based global sensitivity analysis to active subspaces, we provide a complete analysis
of the links between total Sobol' indices, DGSMs, and activity scores for a broad
class of probability distributions. Our analysis is concluded by a result quantifying
approximation errors incurred due to fixing unimportant parameters, deemed so by 
computing their activity scores.  

\end{itemize}

This article is organized as follows. In section~\ref{sub:ac}, a brief
theoretical background on the active subspace methodology is provided. In
section~\ref{sub:gsa}, it is shown that the activity scores provide a
reasonable approximation to the DGSMs especially in a high-dimensional setting.
Additionally, a relationship between the three global sensitivity measures,
namely, the activity scores, DGSMs, and the total Sobol' indices is
established. In section~\ref{sec:method}, a systematic framework for computing
the active subspace using an iterative gradient-based approach is provided.
Numerical results based on the gradient-free approach are compared with those
obtained using the gradient-based approach.  The active subspace is computed
for the H$_2$/O$_2$ reaction kinetics problem whereby only the $A_i$'s are
considered as uncertain. In this case, the two approaches are observed to yield
consistent results. In section~\ref{sec:app}, the gradient-based approach is
further implemented to compute the active subspace for the high-dimensional
H$_2$/O$_2$ reaction kinetics problem. The resulting active subspace is found
to be 1-dimensional thereby indicating enormous potential for computational
savings in this application. The active subspace is also computed using the
gradient-free approach. It is observed that although the uncertainty in the
ignition delay is reasonably approximated, the gradient-free approach fails to
capture the global sensitivity of all the uncertain inputs. Hence, a trade-off
between efficiency and accuracy between the two approaches is observed in this
case.  Finally, a summary and discussion based on our findings is included in
section~\ref{sec:conc}.










 




