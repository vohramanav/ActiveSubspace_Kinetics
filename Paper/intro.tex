\section{Introduction}
\label{sec:intro}

%Constributions:
%\begin{enumerate}
%\item Iterative computation of active subspaces for efficient surrogate modeling 
%\item Comparison of gradient-free and gradient based approaches.
%\item Perspectives on activity scores as means of approximating DGSMs.
%\item Application to challenging chemical kinetics problem. 
%\end{enumerate}
%Insert introduction here.

% Outline
%1. Significance of the H2/O2 reaction. Include the mechanism and a discussion on why
%   is it important to study the kinetics? 
%2. Literature survey of studies aimed at quantifying the uncertainty associated with
%reaction kinetics (including my own work) and specifically for the H2/O2 reaction.
%3. Focus of this work and associated challenges (include a table).
%4. Key contributions.
%5. Organization

Time evolution of a chemically reacting system is largely dependent upon rate
constants associated with individual reactions. The rate constants are
typically assumed to exhibit a certain correlation with temperature (e.g.
Arrhenius-type). Hence, accurate specification of the rate-controlling
parameters is critical to the fidelity of simulations. However, in practical
applications, the parameters are either specified using expert knowledge or
estimated based on a regression-fit to a set of sparse and noisy
data~\cite{Burnham:1987, Burnham:1988, Vohra:2011, Sarathy:2012}.
Consequently, the underlying uncertainty associated with the parameters and
therefore the simulation is typically ignored. Nevertheless, the inclusion of
uncertainty due to corrupted calibration data, parametric uncertainty,
and the discrepancy between predictions and experiments
due to model inadequacy has gained prominence in recent
years~\cite{Vohra:2014, Vohra:2017, Morrison:2018, Hantouche:2018}. 
However, in complex mechanisms involving a large number of reactions,
characterizing the propagation of uncertainty from a large set of inputs to
the model output becomes remarkably challenging due to associated
computational effort. 
A major focus of this paper pertains to the implementation of a robust
framework that aims to identify `important' directions in the input space
that predominantly captures the variability in the model output. These
directions, referred to as the \textit{active subspaces}~\cite{Constantine:2015},
constitute the dominant
eigenspace of a matrix derived from the gradient information of the model
output with respect to an input. The active subspace methodology thus
focuses on reducing the dimensionality of the problem. 

The application problem considered in this work is the
H$_2$/O$_2$ reaction mechanism~\cite{Yetter:1991}. This mechanism has gained
a lot of attention as a potential source of clean energy
locomotive applications~\cite{Das:1996}, and more recently in fuel cells~\cite{Loges:2008}. 
The mechanism involves 19
reactions including chain reactions, dissociation/recombination reactions, and
formation and consumption of intermediate species. For each reaction in
table~\ref{tab:kinetics}, the reaction rate is assumed to show an Arrhenius
correlation with temperature:
%
\be
k_i(T) = A_iT^{n_i}\exp(-E_{a,i}/RT), 
\label{eq:rate}
\ee
%
where $A_i$ is the pre-exponent, $n_i$ is the index of $T$, $E_{a,i}$ is the
activation energy corresponding to the $i^{th}$ reaction, and $R$ is the
universal gas constant.  

\begin{table}[htbp]
\renewcommand*{\arraystretch}{0.9}
\begin{center}
\begin{tabular}{llll}
\toprule
Reaction \#     & Reaction &&\\
\bottomrule
$\mathcal{R}_1$ & H + O$_2$          & $\rightleftharpoons$ & O + OH \\
$\mathcal{R}_2$ & O + H$_2$          & $\rightleftharpoons$ & H + OH \\
$\mathcal{R}_3$ & H$_2$ + OH         & $\rightleftharpoons$ & H$_2$O + H \\
$\mathcal{R}_4$ & OH + OH            & $\rightleftharpoons$ & O + H$_2$O \\
$\mathcal{R}_5$ & H$_2$ + M          & $\rightleftharpoons$ & H + H + M \\
$\mathcal{R}_6$ & O + O + M          & $\rightleftharpoons$ & O$_2$ + M \\
$\mathcal{R}_7$ & O + H + M          & $\rightleftharpoons$ & OH + M \\
$\mathcal{R}_8$ & H + OH +M          & $\rightleftharpoons$ & H$_2$O + M \\
$\mathcal{R}_9$ & H + O$_2$ + M      & $\rightleftharpoons$ & HO$_2$ + M \\
$\mathcal{R}_{10}$ & HO$_2$ + H      & $\rightleftharpoons$ & H$_2$ + O$_2$ \\
$\mathcal{R}_{11}$ & HO$_2$ + H      & $\rightleftharpoons$ & OH + OH \\
$\mathcal{R}_{12}$ & HO$_2$ + O      & $\rightleftharpoons$ & O$_2$ + OH \\
$\mathcal{R}_{13}$ & HO$_2$ + OH     & $\rightleftharpoons$ & H$_2$O + O$_2$ \\
$\mathcal{R}_{14}$ & HO$_2$ + HO$_2$ & $\rightleftharpoons$ & H$_2$O$_2$ + O$_2$ \\
$\mathcal{R}_{15}$ & H$_2$O$_2$ + M  & $\rightleftharpoons$ & OH + OH + M \\
$\mathcal{R}_{16}$ & H$_2$O$_2$ + H  & $\rightleftharpoons$ & H$_2$O + OH \\
$\mathcal{R}_{17}$ & H$_2$O$_2$ + H  & $\rightleftharpoons$ & HO$_2$ + H$_2$ \\
$\mathcal{R}_{18}$ & H$_2$O$_2$ + O  & $\rightleftharpoons$ & OH + HO$_2$ \\
$\mathcal{R}_{19}$ & H$_2$O$_2$ + OH & $\rightleftharpoons$ & HO$_2$ + H$_2$O \\
\bottomrule
\end{tabular}
\end{center}
\caption{Reaction mechanism for H$_2$/O$_2$ from~\cite{Yetter:1991}}.
\label{tab:kinetics}
\end{table}

The global reaction associated with the H$_2$/O$_2$ mechanism can
be considered as follows:
\be
2\text{H}_2 + \text{O}_2 \rightarrow 2\text{H}_2\text{O}.
\label{eq:global}
\ee 
The equivalence ratio ($\phi$) is given as follows:
%
\be
\phi = \frac{(M_{\text{H}_2}/M_{\text{O}_2})_\text{obs}}{(M_{\text{H}_2}/M_{\text{O}_2})_\text{st}}.
\label{eq:phi}
\ee
%
where the numerator on the right-hand-side denotes the fuel (H$_2$)-oxidizer
(O$_2$) ratio at given condition to the same quantity under stoichiometric
conditions. In this study, computations were performed at fuel-rich conditions,
$\phi$ = 2.0. Homogeneous ignition at constant pressure is simulated using the
TChem software package~\cite{Safta:2011}, and the time required for the rate of
temperature increase to exceed a given threshold, regarded as \emph{ignition delay}
is recorded. 

We are interested in investigating the impact of uncertainty in the
rate-controlling parameters, pre-exponent~($A_i$) and the activation
energy~($E_{a,i}$) on ignition delay. The $A_i$'s are considered to be
uniformly distributed about their nominal estimates provided
in~\cite{Yetter:1991}.  The total number of uncertain inputs is 33 which makes
the present problem computationally challenging. 
To reduce the effort, we explore two
strategies: 
a gradient-based approach and  
a gradient-free approach.
%involving local linear approximation which does notrequire gradient computation
%(grad-free).  
\alennote{Thorough the paper 
change grad-based to gradient-based, and similarly for grad-free}. 
%
These strategies are based on algorithms 1.1 and 1.2 
in~\cite{Constantine:2015}. In this work, an
iterative approach is adopted for implementing the two strategies.
Consequently, the amount of computational effort appears to have decreased
\alennote{This will be unclear to readers}, and
is found to be relatively less in the case of grad-free approach as discussed
later in section~\ref{sec:method}.

\alennote{This paragraph will not be meaningful to readers, especially those not familiar
with active subspaces; Need to explain DGSMs, their importance, 
and say something general about our
approach for approximating them.}
Components of the dominant eigenvectors can be used to estimate the so-called
activity scores~\cite{Diaz:2016,Constantine:2017} which provide an insight into
the relative importance of the uncertain parameters. In section~\ref{sub:gsa},
we demonstrate that the activity scores could be used to approximate the
derivative-based global sensitivity measures (DGSMs)~\cite{Sobol:2009}, and
establish a relationship between the Sobol' total effect index, the DGSMs, and
the activity scores. 

%Below, we summarize key contributions of the paper and its
%organization. 

\paragraph{Contributions}
Main contributions of this paper are as follows: (1) A perspective on activity
scores as means for approximating the DGSMs is provided. (2) A general
relationship between DGSMs, activity scores, and the total Sobol' index is
established. (3) Iterative gradient-based and gradient-free approaches for
computing the active subspace are demonstrated. The two approaches are shown to
be consistent and yield greater computational gains compared to the
conventional approach. As mentioned above, the gains are greater in the case of
grad-free approach. (4) The grad-free approach is implemented for discovering
an active subspace in a 33-dimensional parameter space for the H$_2$/O$_2$
reaction kinetics application. 
\alennote{This says too much. Focus on the actual novel 
contributions; also, the application should be made more prominent.}
 

 \paragraph{Organization}
 The paper is organized as follows. In Section~\ref{sub:ac}, a brief theoretical background on
 the active subspace methodology is provided. In Section~\ref{sub:gsa}, it is shown that the activity
 scores provide a reasonable approximation to the DGSMs especially in a high-dimensional setting. 
 Additionally, a relationship between the three global sensitivity measures, namely, the activity scores,
 DGSMs, and the total Sobol' indices is established. In Section~\ref{sec:method}, a systematic framework
 for computing the active subspace using the grad-based, and the grad-free approaches is presented. 
The two approaches are implemented to the H$_2$/O$_2$ reaction kinetics problem whereby only
the $A_i$'s are considered as uncertain. Our findings illustrate the consistency in the results based on
the two approaches thereby encouraging the use of the grad-free approach. In Section~\ref{sec:app},
we further explore the suitability of the grad-free approach by implementing it to a higher-dimensional
H$_2$/O$_2$ reaction kinetics problem by accounting for the uncertainty in $E_{a,i}$'s as well. 
Finally, we draw important conclusions based on the various aspects of this work in Section~\ref{sec:conc}.










 




