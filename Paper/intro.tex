\section{Introduction}
\label{sec:intro}

%Constributions:
%\begin{enumerate}
%\item Iterative computation of active subspaces for efficient surrogate modeling 
%\item Comparison of gradient-free and gradient based approaches.
%\item Perspectives on activity scores as means of approximating DGSMs.
%\item Application to challenging chemical kinetics problem. 
%\end{enumerate}
%Insert introduction here.

% Outline
%1. Significance of the H2/O2 reaction. Include the mechanism and a discussion on why
%   is it important to study the kinetics? 
%2. Literature survey of studies aimed at quantifying the uncertainty associated with
%reaction kinetics (including my own work) and specifically for the H2/O2 reaction.
%3. Focus of this work and associated challenges (include a table).
%4. Key contributions.
%5. Organization

Time evolution of a chemically reacting system is largely dependent upon rate constants
associated with individual reactions. The rate constants are typically assumed to exhibit
a certain correlation with temperature (e.g. Arrhenius-type). Hence, accurate specification
of the rate-controlling parameters is critical to the fidelity of simulations. However,
in practical applications, the parameters are either specified using expert knowledge or
estimated based on a regression-fit to a set of sparse and noisy data. Consequently, the
underlying uncertainty associated with the parameters and therefore the simulation-prediction
is typically ignored. Vohra et al. calibrated the Arrhenius diffusion parameters to quantify
the uncertainty in mixing rates in Zr/Al reactive multilayers using a Bayesian framework
in~\cite{Vohra:2014} and~\cite{Vohra:2017}. The framework accounted for prior knowledge,
measurement noise, and the discrepancy between model and experiments. Bayesian calibration
was accelerated in these efforts using a polynomial chaos surrogate for the model. However,
the number of uncertain parameters for calibration were limited to two in both cases. In
higher dimensions, the construction of surrogate as well as the calibration process become
remarkably challenging due to increase in computational effort.      

