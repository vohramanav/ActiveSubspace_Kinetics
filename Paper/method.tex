\section{Methodology}
\label{sec:method}

%% Grad-based algorithm

\bigskip
\begin{breakablealgorithm}
\renewcommand{\algorithmicrequire}{\textbf{Input:}}
\renewcommand{\algorithmicensure}{\textbf{Output:}}
  \caption{An iterative gradient-based approach for discovering the active subspace}
  \begin{algorithmic}[1]
\Require $\theta_l$, $\theta_u$, $\tau$. 
\Ensure $\Lambda$, $W$, $\alpha$. 
    \Procedure{Gradient-based}{}
	\State Draw $n_1$ random samples, $\{\bm{\xi}_k\}_{k=1}^{n_1}$ $\in$ [-1,1]
         according to $\bm{f(\xi)}$.
	\State Project to the physical space:
        $\{\bm{\theta}_k\}_{k=1}^{n_1}=\theta_l+0.5(\theta_u-\theta_l)\{\bm{\xi}_k\}_{k=1}^{n_1}$
	\State $N_\text{total}$ = $n_1$ 
	\State Compute $\bm{g}^k = \nabla_{\bm{\theta}}G(\bm\theta_k)$, 
             $k=1, \ldots, N_\text{total}$.  
	\Statex\hspace{5mm} Using Finite Difference:
	\Statex\hspace{5mm} i. Assign a small increment, $d\xi$.
	\Statex\hspace{5mm} ii. Augment the set of samples with neighboring points.
	\be \{\bm{\Xi}_k\}_{k=1}^{n_1(d+1)}:~\{\bm{\xi}_k\}_{k=1}^{n_1} \cup
        \{\xi_{k,j}+d\xi\}_{j=1}^{d} \nonumber
	\ee
	\Statex\hspace{5mm} iii. Project to the physical space:
        $\{\bm{\theta}_k\}_{k=1}^{n_1(d+1)}=\theta_l+0.5(\theta_u-\theta_l)\{\bm{\Xi}_k\}_{k=1}^{n_1(d+1)}$
	\Statex\hspace{5mm} iv. Using the augmented set, $\{\bm{\theta}_k\}_{k=1}^{n_1(d+1)}$
        to compute $\bm{g}^k$. 
	\State Compute the matrix, $\mathcal{C}$ = 
        $\frac{1}{N_\text{total}}\sum\limits_{k=1}^{N_\text{total}}[\bm{g}^k][\bm{g}^k]^\top$
	\State Eigenvalue decomposition, $\mathcal{C}$ = $W^{(0)}\Lambda^{(0)} W^{(0)\top}$
	\State Partition the Eigenpairs: $\Lambda^{(0)}~=~ 
        \begin{bmatrix} \Lambda_1^{(0)} & \\ & \Lambda_2^{(0)} \end{bmatrix}$, 
        $W^{(0)}~=~\begin{bmatrix} w_1^{(0)} & w_2^{(0)} \end{bmatrix}$, 
        $\Lambda_1^{(0)}\in \mathbb{R}^{d\times\mathcal{S}}$
	\State Set $r$ = 0
	\Loop
		\State Set $r$ = $r$ + 1
		\State Draw $n_r$ = $\lceil \beta n_1 \rceil$ new random samples 
                $\{\bm{\xi}_k\}$ $\in$ [-1,1], $k = n_{r-1}+1,\ldots,n_{r-1}+n_r$.
		\State Project $\bm{\xi}_k$~$\rightarrow$~$\bm{\theta}_k$.
		\State $N_\text{total}$ = $N_\text{total}$ + $n_r$ 
		\State Compute $\bm{g}^k = \nabla_{\bm{\theta}}G(\bm\theta_k)$, 
             	$k=n_{r-1}+1, \ldots, n_{r-1}+n_r$.  
		\State Compute $\mathcal{C}$ = 
        	$\frac{1}{N_\text{total}}\sum\limits_{k=1}^{N_\text{total}}[\bm{g}^k][\bm{g}^k]^\top$
		\State Eigenvalue decomposition, $\mathcal{C}$ = $W^{(r)}\Lambda^{(r)} W^{(r)\top}$
		\State Compute $\delta w_{1,j}^{(r)}$ = 
                       \scalebox{1.25}{$\frac{\|w_{1,j}^{r} - 
                       w_{1,j}^{r-1}\|}{\|w_{1,j}^{r-1}\|}$}, 
                       $j = 1,\ldots,\mathcal{S}$.
		\If {$\max\left(\delta w_{1,j}^{(r)}\right)<\tau$}
			\State break
		\EndIf
	\EndLoop
	\State Compute $\alpha_i(N_\text{total}) = \sum\limits_{j=1}^{\mathcal{S}} \Lambda_{1,j}w_{1,j}^2$,
	$i=1,\ldots,d$.
	
    \EndProcedure
  \end{algorithmic}
  \label{alg:screen}
\end{breakablealgorithm}
\bigskip

