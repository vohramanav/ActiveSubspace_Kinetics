\section{Summary and Discussion}
\label{sec:conc}
 
In this work, we focused on the uncertainty associated with the
rate-controlling parameters in the H$_2$/O$_2$ reaction mechanism
\rebut{as well as the initial pressure, temperature, and stoichiometry} and its
impact on ignition delay predictions. The mechanism involves 19 different
reactions and in each case, the reaction rate depends upon the choice of a
pre-exponent and an activation energy. Hence, in theory, the evolution of the
chemical system depends upon 38 rate parameters \rebut{and three initial conditions}.
However, we considered 
epistemic uncertainty in all pre-exponents and activation energies with non-zero
nominal values i.e. a total of 33 rate parameters instead of 38 \rebut{in addition to
the three initial conditions}.  
%Conventional means for uncertainty quantification such as those involving
%surrogate model construction as well as sensitivity analysis would thus be
%computationally challenging.
To facilitate efficient uncertainty analysis, we focused our efforts on
reducing the dimensionality of the problem by identifying important directions
in the parameter space such that the model output 
predominantly varies along these directions. These important directions
constitute the active subspace. Additionally, we demonstrated that the activity scores,
computed using the components of the dominant eigenvectors provide an efficient
means for approximating derivative based global sensitivity measures (DGSMs).
Furthermore, we established generalized mathematical linkages between the
different global sensitivity measures: activity scores, DGSMs, and total Sobol'
index which could be exploited to reduce computational effort associated with
global sensitivity analysis. 
 
Active subspace computation requires repeated evaluations of the gradient of
the QoI i.e. the ignition delay. For this purpose, we explored two approaches,
namely, perturbation and regression. Both approaches were shown to yield
consistent results for the 19-dimensional problem wherein only the
pre-exponents were considered to be uncertain.  \rebut{It was observed that the
computational effort required to obtain a converged active subspace was
comparable for the two approaches. However, the predictive accuracy of the
perturbation approach was found to be relatively higher. Moreover, a
1-dimensional active subspace was shown to reasonably approximate the
uncertainty in the ignition delay.} Additionally, the activity scores were also
shown to be consistent with the screening metric estimates based on DGSMs
in~\cite{Vohra:2018}. An iterative procedure was adopted to enhance the
computational efficiency. 

The active subspace was further computed for a 36-dimensional problem
wherein all pre-exponents and activation energies with non-zero nominal
estimates \rebut{as well as the initial conditions} were considered uncertain. 
\rebut{Once again, consistent results were obtained using the two approaches.
A 1-dimensional active subspace was shown to reasonably
capture the uncertainty in the ignition delay in this case. However, the
computational effort required to compute a converged active subspace
using perturbation was found to be half of the effort required in the case
of regression. Predictive accuracy of the two approaches was found to 
be comparable. Hence, perturbation seems like a preferred approach
for the higher-dimensional problem based on our findings. GSA results indicated
that the variability in the ignition delay is predominantly due to the 
uncertainty in the rate parameters, $A_1$ and $A_9$ with significant
contributions from $A_{15}$, $A_{17}$, and $E_{a,15}$. Additionally, the
ignition delay was found to be sensitive towards $T_0$.}

Based on our findings, the perturbation approach is preferable for active
subspace computation; the computational cost of this approach can be reduced
significantly, if more efficient gradient computation techniques (e.g.,
adjoint-based approaches or automatic differentiation) are feasible. The
regression-based approach can be explored in situations involving intensive
simulations where gradient computation is very challenging. 

%
%and the goal is obtaining rough estimates of the statistics of the QoI as
%opposed to a more detailed analysis such as those involving parametric
%sensitivity.  
%
We also mention that alternate regression-based approaches such as ones based
on computing a global quadratic model have been proposed and used in the
literature; see e.g.,~\cite{Constantine:2017a}.  The applicability of such an
approach in the context of high-dimensional chemical reaction networks is
subject to future work. 

The computational framework presented in this work is agnostic to the choice of
the chemical system and can be easily adapted for other systems as long as the
quantity of interest is continuously differentiable in the considered domain of
the inputs.  We have demonstrated that the active subspace could be exploited
for efficient forward propagation of the uncertainty from inputs to the output.
The resulting activity scores and the low-dimensional surrogate could further
guide optimal allocation of computational resources for calibration of the
important rate-controlling parameters \rebut{and input conditions} in a
Bayesian setting.  Additionally, dimension reduction using active subspaces
could assist in developing robust formulations for predicting discrepancy
between simulations and measurements due to epistemic uncertainty in the model
inputs.
