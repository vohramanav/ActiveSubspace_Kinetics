\section{Introduction}
\label{sec:intro}

%Constributions:

Time evolution of a chemically reacting system is largely dependent upon rate
constants associated with individual reactions. The rate constants are
typically assumed to exhibit a certain correlation with temperature (e.g.,
Arrhenius-type). Hence, accurate specification of the rate-controlling
parameters is critical to the fidelity of simulations. However, in practical
applications, these parameters are either specified using expert knowledge or
estimated based on a regression fit to a set of sparse and noisy
data~\cite{Burnham:1987, Burnham:1988, Vohra:2011, Sarathy:2012}.
%Consequently, the underlying uncertainty associated with the parameters and
%therefore the simulation is typically ignored. 
Intensive research efforts in recent years within the field of uncertainty quantification (UQ)
address the quantification and propagation of uncertainty in system models due to 
inadequate data, parametric uncertainty, and model errors~\cite{Vohra:2014, 
Vohra:2017, Morrison:2018, Hantouche:2018, Nannapaneni:2016, Sankararaman:2015,
Reagana:2003}. 

In complex mechanisms involving a large number of reactions, characterizing the
propagation of uncertainty from a large set of inputs to the model output is
challenging due to the associated computational effort.  A major focus of this
article is the implementation of a robust framework that aims to identify
\emph{important} directions in the input space that predominantly capture the
variability in the model output. These directions, which constitute the so called
\emph{active subspace}~\cite{Constantine:2015}, are the dominant eigenvectors
of a matrix derived from the gradient information of the model output with
respect to an input. The active subspace methodology thus focuses on reducing
the dimensionality of the problem, and hence the computational effort
associated with uncertainty propagation. 

The application problem considered in this work is the
H$_2$/O$_2$ reaction mechanism from~\cite{Yetter:1991}. This mechanism has gained
a lot of attention as a potential source of clean energy for
locomotive applications~\cite{Das:1996}, and more recently in fuel
 cells~\cite{Loges:2008,Cosnier:2016}. 
The mechanism involves 19
reactions including chain reactions, dissociation/recombination reactions, and
formation and consumption of intermediate species; see Table~\ref{tab:kinetics}. 
For each reaction, the reaction rate is assumed to follow an Arrhenius
correlation with temperature:
%
\be
k_i(T) = A_iT^{n_i}\exp(-E_{a,i}/RT), 
\label{eq:rate}
\ee
%
where $A_i$ is the pre-exponent, $n_i$ is the \rebut{temperature
exponent}, $E_{a,i}$ is the
activation energy corresponding to the $i^{th}$ reaction, and $R$ is the
universal gas constant.  \rebut{The Arrhenius rate law in~\eqref{eq:rate} 
is often interpreted in a logarithmic form as follows:
%
\be
\log(k_i) = \log(A_i) + n_i\log(T) - E_{a,i}/RT. 
\label{eq:ratelog}
\ee
}
%
%
\begin{table}[htbp]
\renewcommand*{\arraystretch}{0.9}
\begin{center}
\begin{tabular}{llll}
\toprule
Reaction \#     & Reaction &&\\
\bottomrule
$\mathcal{R}_1$ & H + O$_2$          & $\rightleftharpoons$ & O + OH \\
$\mathcal{R}_2$ & O + H$_2$          & $\rightleftharpoons$ & H + OH \\
$\mathcal{R}_3$ & H$_2$ + OH         & $\rightleftharpoons$ & H$_2$O + H \\
$\mathcal{R}_4$ & OH + OH            & $\rightleftharpoons$ & O + H$_2$O \\
$\mathcal{R}_5$ & H$_2$ + M          & $\rightleftharpoons$ & H + H + M \\
$\mathcal{R}_6$ & O + O + M          & $\rightleftharpoons$ & O$_2$ + M \\
$\mathcal{R}_7$ & O + H + M          & $\rightleftharpoons$ & OH + M \\
$\mathcal{R}_8$ & H + OH +M          & $\rightleftharpoons$ & H$_2$O + M \\
$\mathcal{R}_9$ & H + O$_2$ + M      & $\rightleftharpoons$ & HO$_2$ + M \\
$\mathcal{R}_{10}$ & HO$_2$ + H      & $\rightleftharpoons$ & H$_2$ + O$_2$ \\
$\mathcal{R}_{11}$ & HO$_2$ + H      & $\rightleftharpoons$ & OH + OH \\
$\mathcal{R}_{12}$ & HO$_2$ + O      & $\rightleftharpoons$ & O$_2$ + OH \\
$\mathcal{R}_{13}$ & HO$_2$ + OH     & $\rightleftharpoons$ & H$_2$O + O$_2$ \\
$\mathcal{R}_{14}$ & HO$_2$ + HO$_2$ & $\rightleftharpoons$ & H$_2$O$_2$ + O$_2$ \\
$\mathcal{R}_{15}$ & H$_2$O$_2$ + M  & $\rightleftharpoons$ & OH + OH + M \\
$\mathcal{R}_{16}$ & H$_2$O$_2$ + H  & $\rightleftharpoons$ & H$_2$O + OH \\
$\mathcal{R}_{17}$ & H$_2$O$_2$ + H  & $\rightleftharpoons$ & HO$_2$ + H$_2$ \\
$\mathcal{R}_{18}$ & H$_2$O$_2$ + O  & $\rightleftharpoons$ & OH + HO$_2$ \\
$\mathcal{R}_{19}$ & H$_2$O$_2$ + OH & $\rightleftharpoons$ & HO$_2$ + H$_2$O \\
\bottomrule
\end{tabular}
\end{center}
\caption{Reaction mechanism for H$_2$/O$_2$ from~\cite{Yetter:1991}}.
\label{tab:kinetics}
\end{table}
%
The global reaction associated with the H$_2$/O$_2$ mechanism can
be considered as follows:
\be
2\text{H}_2 + \text{O}_2 \rightarrow 2\text{H}_2\text{O}.
\label{eq:global}
\ee 
The equivalence ratio ($\Phi$) is given as follows:
%
\be
\Phi = \frac{(M_{\text{H}_2}/M_{\text{O}_2})_\text{obs}}{(M_{\text{H}_2}/M_{\text{O}_2})_\text{st}}.
\label{eq:phi}
\ee
%
where the numerator on the right-hand-side denotes the ratio of the fuel (H$_2$)
and oxidizer (O$_2$) at a given condition to the same quantity under stoichiometric
conditions. In this study, computations were performed at fuel-rich conditions,
$\Phi$ = 2.0. Homogeneous ignition at constant pressure is simulated using the
TChem software package~\cite{Safta:2011} \rebut{using an initial pressure, $P_0$ = 1 atm and
initial temperature, $T_0$ = 900~K.} The time required for the rate of
temperature increase to exceed a given threshold, regarded as \emph{ignition delay}
is recorded. 

We seek to understand the impact of uncertainty in the
rate-controlling parameters, pre-exponent~($A_i$) and the activation
energy~($E_{a,i}$) \rebut{as well as the initial pressure, temperature, and the
equivalence ratio} on the ignition delay. The $A_i$'s associated with all
reactions and the $E_{a,i}$'s with non-zero nominal estimates
are considered to be uniformly distributed about their nominal estimates provided
in~\cite{Yetter:1991}. \rebut{The initial conditions are also considered to be uniformly
distributed about their respective aforementioned values.} 
The total number of uncertain inputs is \rebut{36} which makes
the present problem computationally challenging due to the large number of 
uncertain parameters \rebut{in addition to the initial conditions}.  
To address this challenge, we focus on reducing the dimensionality
of the problem by computing the active subspace.
This involves repeated evaluations of the gradient of a model output with
respect to the input parameters. Several numerical techniques are available
for computing the gradient. Perturbation-based techniques
include finite differences and more advanced methods involving
adjoints~\cite{Jameson:1988,Borzi:2011,Alexanderian:2017}. The
adjoint-based method requires a solution of the state equation (forward solve)
and the corresponding adjoint equation. Hence, it is 
limited by the availability of an adjoint solver. Additional model evaluations
at neighboring points are required if finite difference is used which increases
the computational effort. Regression-based techniques on the other hand,
aim to estimate the
gradient by approximating the model output using a regression fit. These
are computationally less intensive than the former. However, as expected,
there is a trade-off between computational effort and accuracy in the two
approaches for estimating the gradient. 

In this work, we adopt an iterative strategy to reduce the computational effort
associated with active subspace computation. Moreover, we explore two approaches
for estimating the gradient of the ignition delay with respect to the uncertain
rate-controlling parameters: pre-exponent, $A_i$ and the activation energy, $E_{a,i}$,
for $i^{\text{th}}$ reaction in the H$_2$/O$_2$ mechanism \rebut{as well as
the initial conditions: $P_0$, $T_0$, and $\Phi_0$. Note that the equivalence ratio
corresponding to the initial molar ratios of $H_2$ and O$_2$ is denotes as $\Phi_0$.}
As mentioned above, the
first approach uses finite difference to estimate the gradient and is regarded as
the perturbation approach, whereas, the second approach involves a linear
regression-fit to the available set of model evaluations and is regarded as
the regression approach. 

An alternate strategy to dimension reduction involves computing the global
sensitivity measures associated with the uncertain inputs of a model. Depending
upon the estimates of the sensitivity measures, only the important inputs are
varied for the purpose of uncertainty quantification (UQ). Sobol' indices are
commonly used as global sensitivity measures~\cite{Sobol:2001}. They are
used to quantify the relative contributions of the uncertain inputs to the variance
in the output, either individually, or in combination with other inputs. 
Multiple efforts have focused on efficient computation of the Sobol' 
indices~\cite{Sudret:2008,Plischke:2013,Tissot:2015,Li:2016} including the 
derivative-based global sensitivity measures~(DGSMs), developed to
compute approximate upper bounds for the Sobol' indices with much fewer
computations~\cite{Sobol:2009, Lamboni:2013}. It was noted
in~\cite{Diaz:2016,Constantine:2017} that DGSMs can be approximated by
exploiting their links with active subspaces. This led to the definition of the 
so-called \emph{activity scores}. In Section~\ref{sub:gsa}, we build on these
ideas to provide a complete analysis of links between Sobol indices, DGSMs, and
activity scores for functions of independent random inputs whose distribution
law belongs to the class of Boltzmann probability measures. 
It is worth mentioning that computing global sensitivity measures provides 
important information about a model that go beyond dimension reduction. By 
identifying parameters with significant impact on the model output, we can assess
regimes of validity of the model formulation, and gain critical insight into the
underlying physics in many cases. 

The main contributions of this paper are as follows: 
\begin{itemize}
\item 
Active subspace
discovery in a high-dimensional H$_2$/O$_2$ kinetics problem involving \rebut{36}
uncertain inputs: The methodology presented in this work
\rebut{successfully demonstrated that a 1-dimensional active subspace can reasonably 
approximate the uncertainty in the QoI}, indicating immense potential
for computational savings. The presented
analysis can also guide practitioners in other problems of chemical kinetics on using the
method of active subspaces to achieve efficiency in uncertainty propagation.  
\item Comprehensive numerical investigation of the perturbation and the regression approaches: 
We investigate the suitability of both approaches
for estimating the gradient of ignition delay in the H$_2$/O$_2$ mechanism.
Specifically, we compare resulting
active subspaces, surrogate models, and the ability to approximate global
sensitivity measures through a comprehensive set of numerical experiments. Our
results reveal insight into the merits of the methods as well as
their shortcomings.  
\item Analysis of the links between  
global sensitivity measures: 
By connecting the recent theoretical advances in variance-based and
derivative-based global sensitivity analysis to active subspaces, we provide a complete analysis
of the links between total Sobol' indices, DGSMs, and activity scores for a broad
class of probability distributions. Our analysis is concluded by a result quantifying
approximation errors incurred due to fixing unimportant parameters, deemed so by 
computing their activity scores.  

\end{itemize}

This article is organized as follows. In section~\ref{sub:ac}, a brief
theoretical background on the active subspace methodology is provided. In
section~\ref{sub:gsa}, it is shown that the activity scores provide a
reasonable approximation to the DGSMs especially in a high-dimensional setting.
Additionally, a relationship between the three global sensitivity measures,
namely, the activity scores, DGSMs, and the total Sobol' indices is
established. In section~\ref{sec:method}, a systematic framework for computing
the active subspace is provided. 
Numerical results based on the perturbation approach are compared with those
obtained using the regression approach.  The active 
subspace is initially computed for a 19-dimensional 
H$_2$/O$_2$ reaction kinetics problem wherein only the $A_i$'s are
considered as uncertain. We further compute the active subspace
for a 36-dimensional H$_2$/O$_2$ reaction kinetics problem in section~\ref{sec:app}.
\rebut{For both settings, the convergence characteristics and the predictive accuracy
of the two approaches is compared for a given amount of computational effort. The two
approaches are observed to yield consistent results, and a 1-dimensional active subspace
is observed to capture the uncertainty in the ignition delay.}
Finally, a summary and discussion based on our findings is included in
section~\ref{sec:conc}.










 




