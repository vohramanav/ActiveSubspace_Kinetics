\documentclass[11pt,final]{article}
\renewcommand*\familydefault{\sfdefault}
%\usepackage{amssymb,amsmath,amsfonts,comment}
%\usepackage{amsmath,amssymb,graphicx,subfigure,psfrag}
\usepackage{amsmath,amssymb,graphicx,subfigure,psfrag,upgreek}
\usepackage{algorithm,algorithmic}
\usepackage{amssymb,mathrsfs}
\usepackage[margin=1in]{geometry}
\usepackage{graphicx}
\usepackage{color,pdfcolmk}
\usepackage{enumitem,kantlipsum}
\newcommand{\todo}[1]{\noindent\emph{\textcolor{red}{Todo: #1\:}}}
%\newcommand{\alennote}[1]{\noindent\emph{\vspace{1ex}\textcolor{cyan}{Alen: #1\:}}\\[1ex]}
\newcommand{\referee}[1]{\vspace{.1ex}\noindent{\textcolor{blue}{#1}}}



\begin{document}

%We thank the reviewers for their careful reading of our article 
%and the helpful comments and suggestions.
%Please find below point-by-point replies (in black) to your comments and
%questions (which are reprinted in blue). To give you an overview of all the
%changes in the paper, we also provide a diff-document that highlights the
%changes between the initial submission and this re-submission.\\[1ex]
\begin{center}
{\bf Summary of Modifications to CNF-D-18-00757}\\[6pt]
{\bf Subspace-based dimension reduction for chemical kinetics applications with 
epistemic uncertainty}\\[6pt]
By \\
Manav Vohra, Alen Alexanderian, Hayley Guy, Sankaran Mahadevan 
\end{center}

%\baselineskip=22pt


\vspace*{1in}

%We thank the reviewers for their assessment of our manuscript. Please find
%below point-by-point replies (in black) to your comments and questions
%(reprinted in blue). Where possible, key modifications have been highlighted
%in blue in the revised manuscript. We sincerely hope that with these
%modifications, the paper is found suitable for publication in the
%{\it Journal of Scientific Computing}.

We are extremely thankful to the reviewers for recognizing potential in this effort and 
for their constructive feedback on the manuscript. A point-by-point
response to the comments has been provided below. Corresponding modifications have been
highlighted (in blue) in the revised manuscript. Major changes in the revised manuscript
are summarized as follows:

\begin{itemize}

\item The complexity of the problem has been increased substantially in two ways: First, 
the non-linearity of the system response has been significantly increased by increasing
the perturbations associated with the uncertain rate-controlling parameters: pre-exponent
and the activation energy. Second, the dimensionality of the problem has also been increased
significantly. Specifically, in addition to the rate parameters, the impact of
perturbing the initial conditions: pressure, temperature, and stoichiometry pertaining to the
H$_2$/O$_2$ reaction mechanism on the ignition delay is investigated. Such analysis using the
proposed framework would provide useful insights to combustion scientists interested in
kinetics applications. Moreover, it illustrates the applicability and robustness of the 
framework to complex kinetics applications. 

\item The two approaches, perturbation and regression, for estimating the model gradient
with respect to uncertain parameters and inputs are compared for convergence behavior and
predictive accuracy using the same amount of
computational effort in the revised manuscript, as suggested by one of the reviewers. 

\end{itemize} 

We sincerely hope that based on the modifications in the revised manuscript and our 
response to reviewers' comments, the paper is found suitable for publication in
{\it Combustion and Flame}. 

\clearpage


\section*{Reviewer \#1}

\begin{enumerate}[wide, labelwidth=!, labelindent=0pt]
\item \referee{The authors find that the uncertainty in the ignition delay time is governed
by the uncertainty in H+O~2$<$=$>$~O+OH.
The connection between this reaction and the ignition delay time is so overwhelming that it
is probably responsible for the 1-dimensional active subspace reduction. Once the sensitivity
indices are computed, the one associated with this key reaction is largest.
To a combustion scientist, the fact that ignition delay time (and the uncertainty in the 
ignition delay time when pre-exponentials or activation energies are changed) is governed 
by H+O2~$<$=$>$~O+OH is obvious. This is THE key reaction in combustion kinetics with 10s 
(if not 100s) of studies dedicated to it (from the beginning of the field…). This is textbook k knowledge.
E.g. Figure 4 in Hong et al. (PCI 33, 2011, p. 309-316) shows that, indeed, this is the key 
reaction in sensitivity (by a large margin…). That figure is a "sensitivity analysis" 
which one can accomplish easily.
So, if this is the case, is the authors' reduction framework working because the nonlinear system
is really "simple" in its response to varying parameters}

\noindent The complexity of the H$_2$/O$_2$ reaction kinetics application in the revised manuscript has been significantly
enhanced by increasing the degree of uncertainty in the rate parameters, and considering additional 
uncertainty in the
initial pressure, temperature, and stoichiometry of the system. As a result, the non-linearity in the system
response is observed to increase substantially as seen in Figures 3 and 6 unlike a straight line as 
observed earlier. Our sensitivity analysis results in Figure 4 using the proposed framework based on the
active subspace methodology are shown to be consistent with our earlier results using the derivative-based
global sensitivity analysis (Ref. 35). The sensitivity results for the higher-dimensional problem in the
revised manuscript reveal that the ignition delay is largely impacted by
the uncertainty in the pre-exponents associated with reactions 1, 9, and the initial temperature. Additionally,
we observe significant sensitivity towards the rate parameters associated with reaction 15 and the pre-exponent
associated with the rate of reaction 17. 
Therefore, depending upon the considered prior uncertainty in the individual rate parameters for the 19
reactions, and the range of values of the initial conditions, the proposed framework helps identify the
key contributors to the uncertainty in the ignition delay. The analysis accounts for individual 
contributions of the parameters as well as contributions due to their interactions with other parameters to the 
output uncertainty in the system response. Through this relatively more complex
problem, we have demonstrated the robustness of the proposed framework. 

As mentioned in Section 6
(Summary and Discussion), the proposed framework is agnostic to the choice of the system and could be extended
to other kinetics applications as long as the system response is continuously differentiable in the domain of
the inputs and is therefore applicable to a wide variety of applications.    

\item \referee{Related to 1) above, it is of paramount importance that the authors consider, at least, 
methane, for which I expect less of a spectral gap in the matrix C. Ideally, more complex fuels with 
more nuanced responses.
In other words, if the method works well for a simple case, what is its value to the community?
The authors must try this out on a much more complex hydrocarbon/oxygen system.}

\noindent  A more complex problem as suggested by the reviewer would manifest itself into an increased number of
uncertain parameters and/or an increased amount of non-linearity in the system response. Therefore, while
we did not implement the framework to a new system, we have significantly increased the non-linearity of the
response by increasing the degree of uncertainty in the rate parameters. In addition to the rate parameters,
we perturb the initial conditions to investigate their impact on the response individually as well as due to
their interaction with the rate parameters. Hence, we have also increased the dimensionality of the problem.
As expected by the reviewer, the resulting spectral gap between the first two eigenvalues is observed to
decrease for this relative more complex scenario as shown in Figure~1 in the revised manuscript. 
The proposed framework has been used effectively to compute the active subspace, construct a low-dimensional
surrogate, and perform a global sensitivity analysis in this case. 

\item \referee{The authors report ignition delay times of 0.1 s (?). See for example Fig. 7 and Tab. 2.
Are these seconds (s)? If this is the case, what kind of conditions (temperature/pressure/stoichiometry)
are the authors considering? They seem very unphysical if the yield an ignition delay time of O(0.1 s).
I could not find any clarification of the initial conditions for the ignition calculations.
This is again a symptom of the disconnect between the authors' aims (a novel method, potentially very useful to the combustion community) and the combustion community's aims (more understanding and tools that can be of help in combustion kinetics).}

\noindent  Depending upon the initial conditions, the ignition delay can range from $\mathcal{O}(10^{-3})$
to $\mathcal{O}(10^3)$ seconds for hydrogen-air mixtures as shown by Starostin et al. in~\cite{Starostin:2011}.
In this work, the initial pressure ($P_0$), temperature ($T_0$), and
the equivalence ratio~($\Phi_0$) considered in the computations are 1~atm, 900~K, and 2.0
respectively. These values have been reported in the revised manuscript in the beginning of
Section 5 on page 15.
In this revision, we perturb the initial conditions to study their impact
on ignition delay estimates. The mean estimate of the ignition delay using data at 10$^4$ random samples in
the 36-dimensional space is found to be approximately 0.15 as reported in Table 2 on page 18 in the
revised manuscript. As mentioned in the paper, we have used 
the TChem software for model predictions and based on our recent communication with the developers at
Sandia National Laboratories, these estimates seem reasonable. 

\item \referee{p.3, line 36. I would call this the ``temperature exponent".}

\noindent The reviewer's suggestion has been incorporated in the revised manuscript on Page 3.

\item \referee{p.6, line 32. This is a spectral gap. Need to give credit to the CSP community, 
although they consider spectral gaps in the nonlinear system, while here the authors consider spectral 
gaps in C, which is related to the system's response to varying model parameters.}

\item \referee{p. 16, Fig. 6. y-axis label. Probably $\lambda_i/\lambda_0$ as goes to 1 for i=1? Typo?}

It is in fact $\log(\lambda_i/\lambda_0)$. The typo has been corrected.


%\noindent Minor comments by the reviewer have also been addressed in the revised manuscript.

\end{enumerate}

\section*{Reviewer \#2}

\begin{enumerate}[wide, labelwidth=!, labelindent=0pt]
\item \referee{There is a lack of quantitative discussion of the expense of regression-vs-perturbation,
in terms of the actual number of model simulations required. The work seems to suggest that perturbation
works better than regression, but then it is a direct and clear consequence of the fact that local 
approximation is supposed to be better than a global one. Ideally, much more quantitative comparison
needs to be made here before reaching any meaningful conclusions.}

\noindent  As suggested by the reviewer, we have focused on a quantitative comparison in terms of the
number of samples and associated model runs required to obtain a converged active subspace
for the two approaches. In addition to convergence characteristics, we have also compared
the two approaches for their predictive accuracy for the same computational effort for the
19-dimensional problem as well as the 36-dimensional problem. In the revised work,
we have increased the dimensionality of
the higher-dimensional case from 33 to 36. Specifically, we have enhanced our analysis
to study the impact of perturbations in the initial conditions: pressure, temperature, and
stoichiometry on the system response i.e. ignition delay.

\item \referee{Another fundamental flaw is related to the additive treatment of the uncertainty of 
the pre-exponential factor A. In most works dealing with uncertainty in chemical kinetics, the
uncertainty form is multiplicative or, equivalently, log(A) is the underlying fundamental parameter 
with additive uncertainty range. In the current form of the uncertainty, the problem is too benign,
and barely carries any practical usability for the practitioners.}

\noindent The reviewer has made a valuable suggestion. Therefore, in the revised manuscript, we have
updated the computations for the active subspace using perturbation and regression by considering
the uncertainty in $\log(A)$ as opposed to $A$. As a result, the non-linearity of the system
response is observed to increase significantly. As shown in Figures~3 and 6, 
the variability of the ignition delay in the
active subspace is no longer a straight line as observed earlier. Hence, we have substantially
increased the complexity of the problem by increasing the non-linearity in the system response
as well as its dimensionality to illustrate the robustness and the applicability of the proposed framework
for a wide range of kinetics applications. 

\item \referee{Why not use the word 'active' in the title, since active subspace is essentially
what has been done here?}

\noindent  The title has been updated as ``Active subspace-based dimension reduction for chemical kinetics
applications with epistemic uncertainty".

\item \referee{3.36 $n_i$ is index or power? Is this an established term for the power of T? 
Also, what are the values of $n_i$'s?}

\noindent  In the revised manuscript, we denote $n_i$ as the `temperature exponent' as suggested by
the other reviewer. Yes, it is an established quantity and its value in addition to the
rate constants for the 19 reactions
in the H$_2$/O$_2$ mechanism is taken from [Yetter et. al., 1991] and implemented in the
TChem software used for model predictions in this work. We have clarified this on Page 4
in the revised manuscript. 

\item \referee{3.51 'rate of temperature' or 'temperature'?}

\noindent  The ignition delay is considered as the time required for the {\it rate of temperature}
increase to exceed a threshold, in this work.

\item \referee{4.57 not clear where the number 33 comes from since we have 19 reactions. I see it is explained at the beginning of Section 5, but it should be noted here briefly, too.}

\noindent  We could not locate the number 33 in the location specified by the reviewer. Nevertheless, we would like
to point out that on Page 4 (just below Table 1) in the introduction, 
it is mentioned that the activation energies with a non-zero nominal value are considered as uncertain. 
Since the uncertainty in the activation energy is considered only in the higher-dimensional case
presented in Section 5, we have
specified the reactions for which it is 0 in the beginning of that section.

\item \referee{6.19 distribution-$>$density function}

\noindent  The revised manuscript has been updated to include the reviewer's suggestion. 

\item \referee{6.38: typo, $w_r$ should be $w_1$}

\noindent  The typo has been fixed in the revised manuscript. 

\item \referee{7.23-27: why 'statistical independence' and not just 'independence'? Is there any other 
type of independence?}

\item \referee{7.29-30: Any PDF can be written in a Boltzmann form, V(x) is just the negative log-PDF, 
I am not sure if it is necessary to bring up a Boltzmann distribution language here.
Also, it could be confused with Maxwell-Boltzmann distribution, which is a very specific PDF form.}


\item \referee{7.40: Explain what E and V are with respect to in Eq (8).}

\noindent  The equation as well as the following description of the expectation (E) and the variance (V)
has been updated to address the reviewer's comment on Page~7 in the revised manuscript. 

\item \referee{7.50: Just say in words that it is the i-th diagonal element of matrix C.}

\noindent Following the reviewer's suggestion, we have updated the text in the revised manuscript to
clearly note that the DGSM $\nu_i(f)$ is the $i^{\text{th}}$ diagonal element in the matrix, $\mathbf{C}$
on Page 8.

\item \referee{10.56: what if there is no eigenvalue gap of O(100)?}

\noindent A large gap in the eigenvalue spectrum typically indicates that a low-dimensional (1--2) active subspace 
would sufficiently capture the variability in the model output due to uncertainty in the inputs. However,
in the revised manuscript, we have shown that the variability in the ignition delay is reasonably
approximated with a 1-dimensional active subspace even though the gap in the first two eigenvalues
is found to be O(10). Hence, a smaller gap in the eigenvalue spectrum might still lead to a reasonable
accuracy with a low-dimensional active subspace depending upon the relationship between the
system response and the inputs. However, it is also possible that we might need to a consider a higher 
dimensional active subspace to be able to approximate the variability in the response with reasonable
accuracy in situations where the gap is small. 

\item \referee{11.33: Alg. 1: should have $\beta$ as input, too?}

\noindent The reviewer is correct. Algorithm 1 in the revised manuscript has been updated to include $\beta$
as an input. 

\item \referee{11.42: Which norm is used? Any intuition why squares of W are used, and not just W?}

\noindent We have used the L$^2$ norm to estimate the relative change in the squared values of corresponding
eigenvector components between successive iterations. The algorithm in the revised manuscript has been updated 
to specify the norm. 
We consider square of individual components due to the fact the both W and -W contain legitimate eigenvectors 
of C. Alternatively, we could have also used the modulus value of the components to assess convergence. 

\item \referee{11.45: what is max over?}

\noindent The max is over all eigenvectors in the active subspace denoted by $j$. This has been specified in
the updated algorithm in the revised manuscript.

\item \referee{12.29: 'higher-order' should be replaced with higher-index or smaller-magnitude, to avoid confusion.}

\noindent The reviewer's suggestion has been incorporated in the revised manuscript.

\item \referee{Figs 2a, 3 and 4 demonstrate the regression-based gradient computation is working
similarly to the perturbation based one. However, the comparison is really qualitative and does
 not provide much insight. Clearly, regression will approximate the gradient less accurately then
a local perturbation, but the latter has a price associated with it as new simulations are needed
(a factor of d+1). It would have been great to see how accurate each method is for comparable
expense, or vice-verse, how expensive each approach is for a comparable accuracy. I know there
is a statement about this at the end of Section 4, but it is neither sufficient nor quantitative.
Same goes for statements in Section 5.2 - it was not clear whether regression- and
perturbation- based approaches had comparable expenses (in terms of model evaluations) or not.}

\noindent Based on the reviewer's feedback, in the revised manuscript, the analysis has been
updated to
compare the perturbation and regression approaches for their convergence and predictive 
accuracy using the same amount of computational effort. As a result, the comparative 
analysis is much more conclusive with regards to the performance of the two approaches. 

\item \referee{For Figure 8b, the reasoning why E$_15$ was not captured by the regression approach,
is not satisfactory. I think the authors could have gone a bit deeper in order to explain this
glaring miss of the regression-based approach. There is some discussion in Section 6, but it is
equally unsatisfactory.}

\noindent Our latest analysis in the revised manuscript has revealed that the two approaches
in fact yield consistent results
for sensitivity analysis even for the 36-dimensional case with increased non-linearity
when using the same amount of
computational effort or the same number of model evaluations. Earlier, the two approaches were
compared for the same number of samples, N. However, the number of model runs in the case of
perturbation was N(d+1); d is the number of uncertain parameters and inputs to the model. Whereas,
the number of model runs in the case of regression was only N. Therefore, it can be said that
the accuracy of the regression-based approximation of the gradient can be increased to the 
extent of perturbation-based approach by increasing the sample size. 

\item \referee{14.4-8: the near-linearity simply makes the problem extremely benign, nearly any
reasonable method would have captured such behavior. It would have been great to consider more
complicated case (more than simply going from 19 to 33 parameters), perhaps looking at logA 
perturbation instead of A would help? Also consider more than 1$\%$ 
perturbations to the input. Otherwise, the problem is too simple and does not do justice to the method.}

\noindent We have made a sincere effort to incorporate these valuable suggestions by the 
reviewer in the revised manuscript. 
Specifically, we perturb $\log(A)$ instead of $A$ which is observed to significantly increase
the non-linearity of the system response. In addition to the rate parameters, we analyze the impact
of perturbing the initial conditions. Moreover, as suggested by the reviewer, we have
increased the magnitude of the perturbations from 1$\%$ to 2$\%$ to illustrate the applicability
and robustness of the proposed approach to more complex scenarios. 

\item \referee{14.8: here and in a few other places, I have an issue with statements like
active subspace is 1-dimensional'. To me, active subspace is as many dimensional as one likes,
so the statement should be changed to something like '1-dimensional active subspace captures
most of the variance/uncertainty/dynamics etc..', or '1-dimensional active subspace provides
very accurate approximation...'.}

\noindent The reviewer's suggestion has been incorporated in the revised manuscript.

\item \referee{14.57: even if nominal is zero, one could perturb them, right? 
Or it is actually indicating the fact that there should be no exponential factor whatsoever.}

\noindent Based on our conversations with the developers of the TChem software at Sandia
National Laboratories, we decided not to perturb the activation energies with a 0
nominal value since the exponential factor is not considered for these reactions, as mentioned
by the reviewer. 

\item \referee{18.35: the remark in the parentheses did not make sense to me.}

\noindent We have provided examples for perturbation-based techniques such as finite difference and
adjoint-based methods for estimating the
gradient of the model output with respect to uncertain parameters and inputs. 

\end{enumerate}


\bibliographystyle{elsarticle-num}
\bibliography{REFER}





































\end{document}
