\documentclass[11pt,final]{article}
\renewcommand*\familydefault{\sfdefault}
%\usepackage{amssymb,amsmath,amsfonts,comment}
%\usepackage{amsmath,amssymb,graphicx,subfigure,psfrag}
\usepackage{amsmath,amssymb,graphicx,subfigure,psfrag,upgreek}
\usepackage{algorithm,algorithmic}
\usepackage{amssymb,mathrsfs}
\usepackage[margin=1in]{geometry}
\usepackage{graphicx}
\usepackage{color,pdfcolmk}
\newcommand{\todo}[1]{\noindent\emph{\textcolor{red}{Todo: #1\:}}}
%\newcommand{\alennote}[1]{\noindent\emph{\vspace{1ex}\textcolor{cyan}{Alen: #1\:}}\\[1ex]}
\newcommand{\referee}[1]{\vspace{.1ex}\noindent{\textcolor{blue}{#1}}}



\begin{document}

%We thank the reviewers for their careful reading of our article 
%and the helpful comments and suggestions.
%Please find below point-by-point replies (in black) to your comments and
%questions (which are reprinted in blue). To give you an overview of all the
%changes in the paper, we also provide a diff-document that highlights the
%changes between the initial submission and this re-submission.\\[1ex]
\begin{center}
{\bf Summary of Modifications to CNF-D-18-00757}\\[6pt]
{\bf Subspace-based dimension reduction for chemical kinetics applications with epistemic uncertainty}\\[6pt]
By \\
Manav Vohra, Alen Alexanderian, Hayley Guy, Sankaran Mahadevan 
\end{center}

\baselineskip=22pt


\vspace*{1in}

%We thank the reviewers for their assessment of our manuscript. Please find
%below point-by-point replies (in black) to your comments and questions
%(reprinted in blue). Where possible, key modifications have been highlighted
%in blue in the revised manuscript. We sincerely hope that with these
%modifications, the paper is found suitable for publication in the
%{\it Journal of Scientific Computing}.

\clearpage


\section{Replies to reviewer \#1}
%\referee{This draft proposed a systematic approach for surrogate model construction in
%reduced input parameter spaces in order to reduce the number of model
%evaluations. The key idea is to approximate the screening parameter iteratively
%in order to identify the unimportant inputs. The screening procedure also
%integrates with the adaptive construction of a surrogate in the reduced space
%to improve the efficiency of the methods. Several numerical examples are used
%to demonstrate the efficiency and accuracy of the proposed framework.}
%
%Comments and Questions:
%\begin{enumerate}
%
%\item \referee{P16, Figure.2,  the estimated screening parameters eq (5) based on eq(1)
%quickly converge with a small number of samples (5-10 samples). It seems that
%if $\mu_i$ is based on eq(1), it will still require a reasonable amount of
%samples, can authors comment on why this example converges with a small number
%of samples? The same question applies to Figure 5 and Figure 7.}
%
%Convergence behavior in this case depends upon the variability in the gradient of
%$G(\theta)$ with respect to individual parameters, $\theta_i$ in their respective
%domains. For the numerical tests and the application considered in this work, the
%gradient is not observed to vary significantly in the considered domain of the
%parameters. In other words, the variance of the estimator is low. 
%Hence, a small number of Monte Carlo samples are able to estimate
%the screening parameters. This is a common observation in many practical
%applications, and therefore the proposed methodology involving an iterative
%approach can yield significant computational savings. 
%
%These points have been added in the discussion in Section~6 (Summary and
%Conclusion, Page 34) in the revised manuscript. 
%
%\end{enumerate}

\section{Replies to reviewer \#2}


\end{document}
