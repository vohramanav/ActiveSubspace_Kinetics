\documentclass[11pt,final]{article}
\renewcommand*\familydefault{\sfdefault}
%\usepackage{amssymb,amsmath,amsfonts,comment}
%\usepackage{amsmath,amssymb,graphicx,subfigure,psfrag}
\usepackage{amsmath,amssymb,graphicx,subfigure,psfrag,upgreek}
\usepackage{algorithm,algorithmic}
\usepackage{amssymb,mathrsfs}
\usepackage[margin=1in]{geometry}
\usepackage{graphicx}
\usepackage{color,pdfcolmk}
\usepackage{enumitem,kantlipsum}
\newcommand{\todo}[1]{\noindent\emph{\textcolor{red}{Todo: #1\:}}}
%\newcommand{\alennote}[1]{\noindent\emph{\vspace{1ex}\textcolor{cyan}{Alen: #1\:}}\\[1ex]}
\newcommand{\referee}[1]{\vspace{.1ex}\noindent{\textcolor{blue}{#1}}}



\begin{document}

%We thank the reviewers for their careful reading of our article 
%and the helpful comments and suggestions.
%Please find below point-by-point replies (in black) to your comments and
%questions (which are reprinted in blue). To give you an overview of all the
%changes in the paper, we also provide a diff-document that highlights the
%changes between the initial submission and this re-submission.\\[1ex]
\begin{center}
{\bf Summary of Modifications to CNF-D-18-00757}\\[6pt]
{\bf Subspace-based dimension reduction for chemical kinetics applications with 
epistemic uncertainty}\\[6pt]
By \\
Manav Vohra, Alen Alexanderian, Hayley Guy, Sankaran Mahadevan 
\end{center}

\baselineskip=22pt


\vspace*{1in}

%We thank the reviewers for their assessment of our manuscript. Please find
%below point-by-point replies (in black) to your comments and questions
%(reprinted in blue). Where possible, key modifications have been highlighted
%in blue in the revised manuscript. We sincerely hope that with these
%modifications, the paper is found suitable for publication in the
%{\it Journal of Scientific Computing}.

\clearpage


\section*{Reviewer \#1}

\begin{enumerate}[wide, labelwidth=!, labelindent=0pt]
\item \referee{The authors find that the uncertainty in the ignition delay time is governed
by the uncertainty in H+O~2$<$=$>$~O+OH.
The connection between this reaction and the ignition delay time is so overwhelming that it
is probably responsible for the 1-dimensional active subspace reduction. Once the sensitivity
indices are computed, the one associated with this key reaction is largest.
To a combustion scientist, the fact that ignition delay time (and the uncertainty in the 
ignition delay time when pre-exponentials or activation energies are changed) is governed 
by H+O2~$<$=$>$~O+OH is obvious. This is THE key reaction in combustion kinetics with 10s 
(if not 100s) of studies dedicated to it (from the beginning of the field…). This is textbook k knowledge.
E.g. Figure 4 in Hong et al. (PCI 33, 2011, p. 309-316) shows that, indeed, this is the key 
reaction in sensitivity (by a large margin…). That figure is a "sensitivity analysis" 
which one can accomplish easily.
So, if this is the case, is the authors' reduction framework working because the nonlinear system
is really "simple" in its response to varying parameters}

The complexity of the H$_2$/O$_2$ reaction kinetics application in the revised manuscript has been significantly
enhanced by increasing the degree of uncertainty in the rate parameters, and considering additional 
uncertainty in the
initial pressure, temperature, and stoichiometry of the system. As a result, the non-linearity in the system
response is observed to increase substantially as seen in Figures 3 and 6 unlike a straight line as 
observed earlier. Our sensitivity analysis results in Figure 4 using the proposed framework based on the
active subspace methodology are shown to be consistent with our earlier results using the derivative-based
global sensitivity analysis (Ref. 35). The sensitivity results for the higher-dimensional problem in the
revised manuscript reveal that the igntion delay is largely impacted by
the uncertainty in the pre-exponents associated with reactions 1, 9, and the initial temperature. Additionally,
we observe significant sensitivity towards the rate parameters associated with reaction 15 and the pre-exponent
associated with the rate of reaction 17. 
Therefore, depending upon the considered prior uncertainty in the individual rate parameters for the 19
reactions, and the range of values of the initial conditions, the proposed framework helps identify the
key contributors to the uncertainty in the ignition delay. The analysis accounts for individual 
contributions of the parameters as well as due to their interactions with other parameters to the 
output uncertainty in the system response. Through this relatively more complex
problem, we have demonstrated the robustness of the proposed framework. 

As mentioned in Section 6
(Summary and Discussion), the proposed framework is agnostic to the choice of the system and could be extended
to other kinetics applications as long as the system response is continuously differentiable in the domain of
the inputs and is therefore applicable to a wide variety of applications.    

\item \referee{Related to 1) above, it is of paramount importance that the authors consider, at least, 
methane, for which I expect less of a spectral gap in the matrix C. Ideally, more complex fuels with 
more nounced responses.
In other words, if the method works well for a simple case, what is its value to the community?
The authors must try this out on a much more complex hydrocarbon/oxygen system.}

A more complex problem as suggested by the reviewer would manifest itself into an increased number of
uncertain parameters and/or an increased amount of non-linearity in the system response. Therefore, while
we did not implement the framework to a new system, we have significantly increased the non-linearity of the
response by increasing the degree of uncertainty in the rate parameters. In addition to the rate parameters,
we perturb the initial conditions to investigate their impact on the response individually as well as due to
their interaction with the rate parameters. Hence, we have also increased the dimensionality of the problem.
As expected by the reviewer, the resulting spectral gap between the first two eigenvalues is observed to
decrease for this relative more complex scenario as shown in Figure~1 in the revised manuscript. 
The proposed framework has been used effectively compute the active subspace, construct a low-dimensional
surrogate, and perform a global sensitivity analysis in this case. 

\item \referee{The authors report ignition delay times of 0.1 s (?). See for example Fig. 7 and Tab. 2.
Are these seconds (s)? If this is the case, what kind of conditions (temperature/pressure/stoichiometry)
are the authors considering? They seem very unphysical if the yield an ignition delay time of O(0.1 s).
I could not find any clarification of the initial conditions for the ignition calculations.
This is again a symptom of the disconnect between the authors' aims (a novel method, potentially very useful to the combustion community) and the combustion community's aims (more understanding and tools that can be of help in combustion kinetics).}

The nominal values of the initial conditions: pressure ($P_0$), temperature ($T_0$), and
the equivalence ratio~($\Phi_0$) considered in the computations are 1~atm, 900~K, and 2.0
respectively. These values have been reported in the revised manuscript in the begining of
Section 5 on page 15.
In this revision, we perturb the initial conditions to study their impact
on ignition delay estimates. The mean estimate of the ignition delay using data at 10$^4$ random samples in
the 36-dimensional space is found to be approximately 0.15 as reported in Table 2 on page 18 in the
revised manuscript. As mentioned in the paper, we have used 
the TChem software for model predictions and based on our recent communication with the developers at
Sandia National Laboratories, these estimates seem reasonable. 

\noindent Minor comments by the reviewer have also been addressed in the revised manuscript.

\end{enumerate}


\end{document}
